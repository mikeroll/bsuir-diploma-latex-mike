\section{Используемые технологии}
\label{sec:techs:intro}

Выбор технологий является важным предварительным этапом разработки сложных информационных систем.
Платформа и язык программирования, на котором будет реализована система, заслуживает большого внимания, так как исследования показали, что выбор языка программирования влияет на производительность труда программистов и качество создаваемого ими кода~\cite[c.~59]{mcconnell_2005}.

Ниже перечислены некоторые факторы, повлиявшие на выбор технологий:
\begin{itemize}
\item Разрабатываемое ПО должно работать на операционных системах Linux и Windows.
\item Программное средство должно быть выполнено в виде клиент-серверного приложения.
\item Дальнейшей поддержкой проекта, возможно, будут заниматься разработчики, не принимавшие участие в выпуске первой версии.
\item Имеющийся разработчик имеет опыт работы с объектно"=ориентированными и с функциональными языками программирования.
\end{itemize}

Основываясь на приведенных факторах, целесообразно выбрать в качестве платформы разработки язык Python. Приняв во внимание необходимость обеспечения доступности дальнейшей поддержки ПО, возможно, другой командой программистов, целесообразно не использовать малоизвестные и сложные языки программирования.

Для реализации поставленной задачи существует необходимость в написании серверного приложения. Эту задачу можно облегчить путём использования набора прикладных библотек (фреймворка) Flask, предназначенного для быстрого прототипирования и реализации веб-приложений различной степени сложности.

Далее проводится характеристика используемых технологий и языков программирования более подробно.



\subsection{Язык программирования Python}
\label{sub:techs:python}
Python — высокоуровневый язык программирования общего назначения, ориентированный на повышение производительности разработчика и читаемости кода. Синтаксис ядра Python минималистичен. В то же время стандартная библиотека включает большой объём полезных функций.

Python поддерживает несколько парадигм программирования, в том числе структурное, объектно-ориентированное, функциональное, императивное и аспектно-ориентированное. Основные архитектурные черты — динамическая типизация, автоматическое управление памятью, полная интроспекция, механизм обработки исключений, поддержка многопоточных вычислений и удобные высокоуровневые структуры данных. Код в Python организовывается в функции и классы, которые могут объединяться в модули (они в свою очередь могут быть объединены в пакеты).

Эталонной реализацией Python является интерпретатор CPython, поддерживающий большинство активно используемых платформ. Он распространяется под свободной лицензией Python Software Foundation License, позволяющей использовать его без ограничений в любых приложениях, включая проприетарные. Есть реализации интерпретаторов для JVM (с возможностью компиляции), MSIL (с возможностью компиляции), LLVM и других. Проект PyPy предлагает реализацию Python на самом Python, что уменьшает затраты на изменения языка и постановку экспериментов над новыми возможностями.

Python — активно развивающийся язык программирования, новые версии (с добавлением/изменением языковых свойств) выходят примерно раз в два с половиной года. Вследствие этого и некоторых других причин на Python отсутствуют стандарт ANSI, ISO или другие официальные стандарты, их роль выполняет CPython.

Отличительные особенности языка Python:
\begin{itemize}
  \item Простой. Python – простой и минималистичный язык. Чтение хорошей программы на Python очень напоминает чтение английского текста. Такая псевдо-кодовая природа Python является одной из его самых сильных сторон. Она позволяет сосредоточиться на решении задачи, а не на самом языке.
  \item Свободный и открытый. Python – это пример свободного и открытого программного обеспечения – FLOSS (Free/Libré and Open Source Software). Пользователь имеет право свободно распространять копии этого программного обеспечения, читать его исходные тексты, вносить изменения, а также использовать его части в своих программах. В основе свободного ПО лежит идея сообщества, которое делится своими знаниями. Это одна из причин, по которым Python так хорош: он был создан и постоянно улучшается сообществом, которое хочет сделать его лучше.
  \item Язык высокого уровня. При написании программы на Python разработчику не нужно отвлекаться на такие низкоуровневые детали, как управление памятью и т.п.
  \item Портируемый. Благодаря своей открытой природе, Python был портирован на множество платформ. Программы на Python могут запускаться на любой из этих платформ без каких-либо изменений (если программа не использует системно-зависимые функции). Python можно использовать в GNU/Linux, Windows, FreeBSD, Macintosh, Solaris, OS/2, Amiga, AROS, AS/400, BeOS, OS/390, z/OS, Palm OS, QNX, VMS, Psion, Acorn RISC OS, VxWorks, PlayStation, Sharp Zaurus, Windows CE и множестве других платформ.
  \item Интерпретируемый. Python не требует компиляции в бинарный код. Программа выполняется из исходного текста. Python сам преобразует этот исходный текст в некоторую промежуточную форму, называемую байткодом, а затем переводит его на машинный язык и запускает. Всё это заметно облегчает использование Python, поскольку нет необходимости заботиться о компиляции программы, подключении и загрузке нужных библиотек и т.д. Вместе с тем, это делает программы на Python намного более переносимыми.
  \item Объектно-ориентированный. Python поддерживает как процедурно-ориентированное, так и объектно-ориентированное программирование. Python предоставляет простые, но мощные средства для ООП, особенно в сравнении с такими большими языками программирования, как C++ или Java.
  \item Расширяемый. Если необходимо добиться очень высокой производительности некоторой части программы или использовать некоторые другие возможности более низкоуровневых языков, можно разработать эту часть программы на C или C++, а затем вызывать её из программы на Python.
  \item Встраиваемый. Python можно встраивать в программы на C/C++, чтобы предоставлять возможности написания сценариев их пользователям.
  \item Обширные библиотеки. Стандартная библиотека Python просто огромна. Она может помочь в решении самыхразнообразных задач, связанных с использованием регулярных выражений, генериро-ванием документации, проверкой блоков кода, распараллеливанием процессов, база-ми данных, веб-браузерами, CGI, FTP, электронной почтой, XML, XML-RPC, HTML, WAV файлами, криптографией, GUI и другими системно-зависимыми вещами. Всё это доступно абсолютно везде, где установлен Python. В этом заключается философия Python <<Всё включено>>.Кроме стандартной библиотеки, существует множество других высококачественных биб-лиотек, доступных в каталоге пакетов.
\end{itemize}

\subsubsection{Объектно-ориентированное программирование. }
Дизайн языка Python построен вокруг объектно-ориентированной модели программирования. Реализация ООП в Python является элегантной, мощной и хорошо продуманной, но вместе с тем достаточно специфической по сравнению с другими объектно-ориентированными языками. Особенности~\cite{wiki_python, byte_of_python}:
\begin{itemize}
\item Классы являются одновременно объектами со всеми ниже приведёнными возможностями.
\item Наследование, в том числе множественное.
\item Полиморфизм (все функции виртуальные).
\item Инкапсуляция (два уровня — общедоступные и скрытые методы и поля). Особенность — скрытые члены доступны для использования и помечены как скрытые лишь особыми именами.
\item Специальные методы, управляющие жизненным циклом объекта: конструкторы, деструкторы, распределители памяти.
\item Перегрузка операторов (всех, кроме is, '.', '=' и символьных логических).
\item Свойства (имитация поля с помощью функций).
\item Управление доступом к полям (эмуляция полей и методов, частичный доступ, и т. п.).
\item Методы для управления наиболее распространёнными операциями (истинностное значение, len(), глубокое копирование, сериализация, итерация по объекту, …)
\item Метапрограммирование (управление созданием классов, триггеры на создание классов, и др.)
\item Классовые и статические методы, классовые поля.
\end{itemize}

\subsubsection{Функциональное программирование. }
Python поддерживает парадигму функционального программирования, в частности:
\begin{itemize}
\item Функция является объектом;
\item Функции высших порядков;
\item Рекурсия;
\item Развитая обработка списков (списочные выражения, операции над последовательностями, итераторы);
\item Аналог замыканий;
\item Частичное применение функции;
\end{itemize}

\subsubsection{Интроспекция. }
Python поддерживает полную интроспекцию времени исполнения. Это означает, что для любого объекта можно получить всю информацию о его внутренней структуре.
Применение интроспекции является важной частью того, что называют pythonic style, и широко применяется в библиотеках и фреймворках Python, таких как PyRO, PLY, Cherry, Django и др., значительно экономя время использующего их программиста.

\subsubsection{Итераторы. }
В программах на Python широко используются итераторы. Цикл for может работать как с последовательностью, так и с итератором. Все коллекции, как правило, предоставляют итератор. Объекты определённого пользователем класса тоже могут быть итераторами. Подробнее об итераторах можно узнать в разделе о функциональном программировании. Модуль itertools стандартной библиотеки содержит много полезных функций для работы с итераторами.

\subsubsection{Генераторы. }
Одной из интересных возможностей языка являются генераторы — функции, сохраняющие внутреннее состояние: значения локальных переменных и текущую инструкцию. Генераторы могут использоваться как итераторы для структур данных и для ленивых вычислений.
При вызове генератора функция немедленно возвращает объект-итератор, который хранит текущую точку исполнения и состояние локальных переменных функции. При запросе следующего значения (посредством метода next(), неявно вызываемого в цикле for) генератор продолжает исполнение функции от предыдущей точки останова до следующего оператора yield или return.

\subsection{Язык описания аппаратуры интегральных схем VHDL}
\label{sub:techs:vhdl}
Возрастающая алгоритмическая сложность аппаратно реализованных устройств приводит к тому, что, как проблемы разработки, описания и применения аппаратуры (hardware), так и подходы к их решению, становятся подобны проблемам и методам решения для современных программных систем (software). Перспективное направление решения этих проблем — применение алгоритмического подхода, создание алгоритмического языка для описания аппаратуры, программирования и структуры, функционирования аппаратных средств обработки информации. Наиболее распространенным языком этого класса, специфицированным международными стандартами, является язык VHDL, который разработан в рамках американского проекта создания нового поколения высокоскоростной элементной базы (Very High Speed Integrated Circuits — VHSIC). Аббревиатура VHDL расшифровывается как VHSIC Hardware Description Language. Расширение языка VHDL — язык VHDL-AMS (Very-High-Speed IС Hardware Description Language — Analog and Mixed Signal) включает также возможности моделирования систем, содержащих и цифровую, и аналоговую части.~\cite{suvorova_vhdl, ivchenko_vhdl}

Язык VHDL предназначен для решения комплекса задач в ходе проектирования и применения цифровых систем, их аппаратных средств, в том числе:
\begin{itemize}
  \item  Описания структуры системы, декомпозиции системы на подсистемы,
спецификации связей и взаимодействия подсистем.
  \item  Спецификации функционирования системы, узлов, блоков, реализуемых
функций. Спецификация дается в алгоритмической форме, с использова-
использованием привычных современному специалисту программных конструкций
алгоритмического языка, включающих в себя спецификацию временного
поведения сигналов и блоков.
  \item  Моделирования системы и ее работы на основе четкой спецификации
структуры системы, а также функционирования ее компонентов.
  \item  Синтеза схемотехнической реализации системы, автоматической генера-
генерации детальной структуры на основе строгой спецификации системы на
языке VHDL — спецификации на более абстрактном уровне.
\end{itemize}

Первоначально язык предназначался для моделирования, но позднее из него было выделено синтезируемое подмножество. Написание модели на синтезируемом подмножестве позволяет автоматический синтез схемы функционально эквивалентной исходной модели. Средствами языка VHDL возможно проектирование на различных уровнях абстракции (поведенческом или алгоритмическом, регистровых передач, структурном), в соответствии с техническим заданием и предпочтениями разработчика. Заложена возможность иерархического проектирования, максимально реализующая себя в экстремально больших проектах с участием большой группы разработчиков. Представляется возможным выделить следующие три составные части языка: алгоритмическую — основанную на языках Ada и Pascal и придающую языку VHDL свойства языков программирования; проблемно ориентированную — в сущности и обращающую VHDL в язык описания аппаратуры; и объектно-ориентированную, интенсивно развиваемую в последнее время.

VHDL является формальной записью, предназначенной для описания функций и логической организации цифровой системы. Функция системы определяется как преобразование значений на входах в значения на выходах. Причем время в этом преобразовании задается явно. Организация системы задается перечнем связанных компонентов.

\subsubsection{Структура VHDL-проекта. }

Объект проекта (entity) представляет собой описание компонента проекта, имеющего четко заданные входы и выходы и выполняющей четко определенную функцию. Объект проекта может представлять всю проектируемую систему, некоторую подсистему, устройство, узел, стойку, плату, кристалл, макроячейку, логический элемент и т.п.

В описании объекта проекта можно использовать компоненты, которые, в свою очередь, могут быть описаны как самостоятельные объекты проекта более низкого уровня. Таким образом, каждый компонент объекта проекта может быть связан с объектом проекта более низкого уровня. В результате такой декомпозиции объекта проекта пользователь строит иерархию объектов проекта, представляющих весь проект в целом и состоящую из нескольких уровней абстракций. Такая совокупность объектов проекта называется иерархией проекта (design hierarchy).Каждый объект проекта состоит, как минимум, из двух различных типов описаний: описания интерфейса и одного или более архитектурных тел.Интерфейс описывается в объявлении объекта проекта  (entity declaration)  и определяет только входы и выходы объекта проекта. Для описания поведения объекта или его структуры служит архитектурное тело (architecture body). Чтобы задать, какие объекты проекта использованы для создания полного проекта, используется объявление конфигурации (configuration declaration).

В языке VHDL  предусмотрен механизм пакетов для часто используемых описаний, констант, типов, сигналов. Эти описания помещаются в объявлении пакета (package declaration).Если пользователь использует нестандартные операции или функции, их интерфейсы описываются в объявлении пакета, а тела содержатся в теле пакета (package body).

Таким образом, при описании цифровой системы на языке VHDL,  пользователь может использовать пять различных типов описаний: объявление объекта проекта, архитектурное тело, объявление конфигурации, объявление пакета и тело пакета. Каждое из описаний является самостоятельной конструкцией языка  VHDL, может быть независимо проанализировано анализатором и поэтому получило название <<Модуль проекта>> (designunit).

Модули проекта, в свою очередь, можно разбить на две категории: первичные и вторичные. К первичным модулям относятся различного типа объявления. Ко вторичным  -  отдельно анализируемые тела первичных модулей. Один или несколько модулей проекта могут быть помещены в один файл, называемый файлом проекта (design file). Каждый проанализированный модуль проекта помещается в библиотеку проекта (design ibrary) и становится библиотечным модулем (library unit). Данная реализация позволяет создать любое число библиотек проекта. Каждая библиотека проекта в языке  VHDL имеет логическое имя (идентификатор). Фактическое имя файла, содержащего эту библиотеку, может совпадать или не совпадать с логическим именем библиотеки проекта. Для ассоциации логического имени библиотеки с соответствующим ей фактическим именем предусмотрен специальный механизм установки внешних ссылок.

По отношению к сеансу работы  VHDL существует два класса библиотек проекта: рабочие библиотеки и библиотеки ресурсов.Рабочая библиотека  -  это библиотека, с которой в данном сеансе работает пользователь и в которую помещается библиотечный модуль, полученный в результате анализа модуля проекта. Библиотека ресурсов  -  это библиотека, содержащая библиотечные модули, ссылка на которые имеется в анализируемом модуле проекта. В каждый конкретный момент пользователь работает с одной рабочей библиотекой и произвольным числом библиотек ресурсов.

Возможность создания и использования многих библиотек ресурсов позволяет пользователю классифицировать библиотечные модули по различным признакам. Например, водной библиотеке хранить описания микросхем одной серии, в другой  -  описания микросхем другой серии и т.д.    Или водной библиотеке хранить описания микросхемс одним типом задержки,  в другой  -  описания микросхем с другим типом задержки и т.д.
