\section{Обзор предметной области}
\label{sec:domain:intro}

В данном разделе будет произведён обзор предметной области задачи, решаемой в рамках дипломного проекта - аутентификации цифровых устройств как механизма защиты интеллектуальной собственности; рассмотрены существующие методы аутентификации цифровых устройств с использованием физически неклонируемых функций, как теоретические, так и применяемые на практике; приведены преимущества и недостатки различных подходов, используемых при проектировании цифровых устройств и систем их аутентификации.
В частности, будут рассмотрены варианты построения аппаратных уникальных неклонируемых идентификаторов цифровых устройств, а так же механизмы и протоколы аутентификации устройств с использованием данных идентификаторов.


\subsection{Интеллектуальная собственность в полупроводниковой промышленности}
\label{sub:domain:ic_ip}
В современной полупроводниковой промышленности процесс проектирования устройств напрямую зависит от поставщиков отдельных модулей, составляющих эти устройства этих устройств. Эти модули рассчитаны на совокупное использование и заранее разрабатываются с учётом возможного применения в устройствах различной степени сложности и предусматривают относительно простое подключение друг к друг.  Данные блоки называются IP-ядрами или IP-блоками (IP-core, IP-block). При создании полупроводниковых устройств системные интеграторы покупают лицензию на IP-ядра для последующего использования в своих продуктах. Различают три основных вида ядер~\cite{counterfeit_ics}:
\begin{itemize}
\item Программные IP-блоки (soft blocks) - блоки, специфицированные  с помощью языка описанию аппаратуры или схожего по уровню абстракции языка, то они представляют собой процессно-инвариантные цифровые IP-блоки, которые могут быть использованы для синтеза на вентильном уровне. Программные блоки очень гибкие и могут быть легко перенесены из одной системы в другую.
\item Физические IP-блоки (hard blocks) - блоки, специфицированные на физическом уровне реализации, например Graphical Database System II (GDSII) для ASIC. Эти блоки обладают фиксированным расположением элементов и предсказуемыми временными характеристиками, энергопотреблением и занимаемой площадью. Их основной недостаток - жесткость дизайна (ограничены одним технологическим процессом) и слабая переносимость.
\item Схемотехнические IP-блоки (firm blocks) - блоки, специфицированные на схемотехническом уровне, без привязки к конкретной топологической реализации. Обычно предоставляются в форме готового списка соединений и более предсказуемы, чем программные блоки. Могут быть легко перенесены на различные технологические процессы и оптизированы для потребностей разрабатывающей стороны.
\end{itemize}

IP-блоки служат базовыми блоками для построения устройств и, таким образом, являются основными объектами интеллектуальной собственности применительно к разработке интегральных микросхем.Закономерным становится вопрос о защите IP-ядер как интеллектуальной собственности от несанкционированного использования, изменения, копирования. Способы защиты этих продуктов, а так же возможный вариант реализации такой системы защиты являются объектом исследования данного дипломного проекта.


\subsection{Физически неклонируемые функции}
\label{sub:domain:pufs}
Физическая криптография, в основе которой лежат физически неклонируемые функции, может найти применение не только в традиционном смысле защиты конфиденциальной информации, но также и в области защиты собственно электронных цифровых устройств от несанкционированного использования, изменения, клонирования. Идея применения физически неклонируемых функций (Phisycal Unclonable Function, PUF, или изначально Physical One-Way Function, POWF) впервые была предложена в работах Р. Паппу (R. Pappu) в 2001 году~\cite{rpappu_powf}, а современное толкование и определение сформулировал П. Туилс (P. Tuyls) в 2006 году~\cite{ptuyls_pufs}. Согласно этому определению, физически неклонируемой функцией называют набор характеристик физической (цифровой) системы, копирование и воспроизведение которого применительно к другим физическим системам невозможно. Цифровые системы, как правило, состоят из множества компонент, физические параметры которых формируются на стадии производства и принимают случайные в некотором диапазоне значения. Контролировать эти значения невозможно в силу хаотичности микро- (а часто и макро-)скопических параметров физической системы, в рамках которой происходит создание устройства. Подобное свойство получило название физической вариации технологического процесса~\cite{ivaniuk_pufs}. Таким образом, уникальность и невоспроизводимость (неклонируемость) цифровой системы обеспечивается наличием подобных случайных параметров. Более того, PUF могут быть реализованы таким образом, который не допускает симуляцию, моделирование или предсказание их поведения. Идея использования PUF в криптографических целях основана на извлечении и использовании этих случайных параметров.

\def\puf_crp {\left\{C_i, R_i\right\}}

PUF - это физическая система, которая при воздействии на неё (запросе) порождает уникальный, но непредсказуемый ответ. Специфический запрос C (Challenge) и соответствующий ему выходной сигнал R (Response) вместе образуют пару запрос-ответ (Challenge-Response Pair, CRP). PUF, в свою очередь, является функцией преобразования множества запросов $ C_i $ в множество ответов $ R_i $:
\begin{equation}
  \label{eq:domain:pufs:puf_main}
  R_i = PUF(C_i)
\end{equation}

По определению У. Рурмаира (U. Ruhrmair), PUF представляют собой сложные неуправляемые физические системы со сверхбольшим объёмом структурной информации (Super High Information Content, SHIC). PUF должны удовлетворять следующим свойствам:
\begin{enumerate}
  \item Физическое копирование системы с сохранением её свойств в виде пар $ \puf_crp $ должно быть практически невозможно.
  \item Информация в виде ответов $ R_i $ на запросы $ C_i $ может быть извлечена из системы многократно и с высокой степенью надёжности.
  \item Множество возможных запросов $ C_i $ должно быть достаточно велико, чтобы получение всех пар $ \puf_crp $ не представлялось возможным за разумное время и количество ресурсов.
  \item Пары $ \puf_crp $ должны быть независимы друг от друга в том смысле, что по известной паре $ \puf_crp $ невозможно получить, смоделировать или предсказать значение любой другой пары или множества пар.
\end{enumerate}


% Виды ФНФ

\subsection{Виды PUF по физической реализации. }
\label{sub:domain:puf_physical_types}


\subsubsection{PUF на оптических элементах. }
\label{sub:domain:puf_physical_types:optical}
Любые физически неклонируемые функции, в которых процесс реакции системы в основе своей имеет какое-либо оптическое явление, называются оптическими.
Как правило, используется объект с неоднородной прозрачностью (подобно светоотражающим пузырькам из примера выше). Именно благодаря рассеивающему действию вещества и вкраплений объект не может быть исследован и смоделирован математически. Оптическое сканирование не может проникнуть вглубь объекта.


\subsubsection{PUF на интегральных микросхемах. }
\label{sub:domain:puf_physical_types:ic}
Логично предположить, что аналогично пузырькам в оптической системе, в иных средах тоже можно получить неклонируемость. И логично рассмотреть среду, и так широко используемую в цифровых устройствах - электрические проводники и диэлектрики.
Любое современное технологичное устройство содержит в себе внушительное количество интегральных микросхем. На них и строится <<электрическая>> PUF. Несмотря на то, что микросхемы изготавливаются по одному и тому же технологическому процессу, каждая из них достаточно уникальна для корректной работы PUF. Это может быть использовано в системах ИБ как уникальный идентификационный признак устройства.
ИМС покрывается слоем защитного вещества со вкраплениями диэлектрика. Эти вкрапления имеют случайные размер и форму. Под этот слой подводятся электроды-датчики. В совокупности с защитным слоем каждый такой электрод обретает свойства конденсатора случайной (зависящей от вкраплений диэлектрика в защитный слой) ёмкости. Очевидно, что случайность ёмкости каждого конденсатора достигается при размере частиц, сравнимом с размером между электродами.


\subsubsection{PUF на полевых транзисторах. }
\label{sub:domain:puf_physical_types:transistors}
В основе таких PUF лежит особенность полевых транзисторов задерживать сигнал, проходящий по нему на непредсказуемое время, зависящее от физических свойств материала транзистора (именно из-за материала такие PUF ещё называют кремниевыми). Физическая система PUF будет состоять из набора пар транзисторов и триггеров-арбитров (см. ниже PUF с арбитром). Триггер будет давать на выход 0 или 1 в зависимости от того, сигнал с какого из пары транзисторов пришёл к нему раньше. Подавая на вход этому набору некоторый сигнал-запрос, на выходе исследователь получит набор значений триггеров - реакцию системы на запрос, уникальную для данного устройства, в состав которого включаются упомянутые транзисторы. Неклонируемость строится на физической неидеальности процесса производства. Изучение функции для последующего математического прогноза реакции на запрос возможно только полным перебором входных сигналов, что является достаточно вычислительно сложной задачей.


\subsubsection{PUF на магнитных элементах. }
\label{sub:domain:puf_physical_types:magnetic}
Практическое применение - уникальный идентификатор магнитной полосы банковской карты. Частицы феррита бария, содержащиеся в пасте-основе магнитной полосы, также имеют случайный размер и форму. Логично сделать вывод, что на случайности их распределения можно построить PUF. Неклонируемость снова зиждется на физической неидеальности процесса производства. Неточности и погрешности не дадут повторить рисунок частиц феррита бария в точности. А математическое моделирование не препятствует выполнению данной PUF её задачи.


\subsection{Виды PUF по принципу работы. }
\label{sub:domain:puf_types}


\subsubsection{PUF типа арбитр (Arbiter PUF). }
\label{sub:domain:puf_types:arbiter}
Физически неклонируемые функции по типу арбитра - разновидность физически неклонируемых функций на основе задержек. Идея состоит в том, чтобы привнести состояние гонки между двумя путями микросхемы. Оба пути заканчиваются элементом-арбитром, который определяет какой из путей был быстрее и выдает соответствующее бинарное значение.


\subsubsection{PUF на базе кольцевого генератора (RO-PUF). }
\label{sub:domain:puf_types:ring_oscillator}
Физически неклонируемые функции на базе кольцевых генераторов также используют неконтролируемые изменения задержек процессов цифровых компонентов в качестве источника случайности. Когда эти компоненты образуют кольцевой генератор, частоты его выходных сигналов различаются, что и используется при формировании бинарного ответа.


\subsubsection{PUF на основе сбоев (Failure PUF). }
Этот тип физически неклонируемых функций основываются на сбоях в поведении комбинаторных логических схем. В идеале у комбинаторной схемы нет внутреннего состояния, что означает, что стационарный выход полностью определяется его входными сигналами. Однако, когда логическое значение на входе изменяется, для достижения стационарного значения на выходе требуется некоторое время. Появление сбоя определяется различиями в задержках различных логических цепей от входов к выходному сигналу. Так как задержки определенных образцов комбинаторных схем вызваны случайными изменениями процесса, появление, количество и форма сбоев выходных сигналов также будут случайными и характерными для определенных образцов схемы. Поэтому оценка поведения сбоев таких схем может быть использована для ответа физически неклонируемой функции.


\subsubsection{PUF на базе SRAM (SRAM PUF). }
\label{sub:domain:puf_types:sram}
Принцип работы физически неклонируемых функций на базе статического оперативного запоминающего устройства (СОЗУ) основан на случайности состояния части ячеек СОЗУ при включении.


\subsubsection{PUF типа бабочка (Butterfly PUF). }
\label{sub:domain:puf_types:arbiter}
Физически неклонируемые функции по типу бабочки имитирует работу ячейки СОЗУ, формируя перекрестные обратные связи. Получается бистабильная схема. Схема принудительно переводится в неустойчивое состояние, после чего схема переходит в одно из двух стабильных состояний, которое зависит от случайной разности задержек в паре линий обратной связи и линии входного сигнала.

Стоит заметить, что в данном списке перечислены лишь базовые варианты реализации PUF. На их основе и на основе комбинаций этих типов может быть построено огромное множество различных сложных PUF. ~\cite{cryptowiki_pufs, rmaes_pufs}


\subsection{Использование PUF для идентификации и аутентификации цифровых устройств}
\label{sub:domain:puf_auth}
Благодаря тому, что физически неклонируемые функции уникальны и сложно воспроизводимы, они могут быть использованы для целей аутентификации. Сценарий аутентификации представлен централизованным проверяющим и объектами (устройствами), чья подлинность проверяется.

Существует два основных подхода к осуществлению системы аутентификации, основанной на физически неклонируемых функциях. Первый состоит в разработке схемы аутентификации, которая напрямую применяет уникальность и непредсказуемость поведения пары запроса-ответа отдельного объекта. Такая схема состоит из двух фаз (регистрации и подтверждения):

Сначала каждый объект проходит регистрацию у проверяющего. Во время этой фазы проверяющий записывает идентификатор каждого объекта и собирает значительное подмножество пар запрос-ответ каждого объекта (устройства) для случайно сгенерированных запросов. Собранные пары запрос-ответ хранятся в базе данных проверяющего.
Во время фазы подтверждения, объект (устройство) посылает проверяющему свой идентификатор. Проверяющий находит его в базе данных, выбирает оттуда случайную пару запрос-ответ, которая соответствует полученному идентификатору. Запрос посылается объекту, объект вычисляет физически неклонируемую функцию и посылает ответ. Проверяющий сравнивает, насколько ответ близок к значению из базы данных, то есть что ответы различаются не более чем на заранее установленное значение. Если проверка прошла успешно, объект аутентифицирован, иначе - аутентификация отклонена. Использованная пара запрос-ответ удаляется из базы данных.
Корректность данной схемы аутентификации обеспечивается тем фактом, что ответы физически неклонируемых функций воспроизводимы самим объектом в течение долгого времени.

Второй подход состоит в получении стойкого и безопасного криптографического ключа из ответа физически неклонируемой функции и использовании этого ключа в каком-либо из существующих криптографических протоколов аутентификации.

При внедрении физически неклонируемой функции в кредитные или смарт-карты, RFID-метки, ценные бумаги и т. д., эти объекты становятся неклонируемыми, а их идентичности проверяемыми с помощью одного из методов, описанных выше.


\subsection{Обзор существующих аналогов}
Подавляющее большинство средств защиты интеллектуальной собственности применительно к цифровым устройствам, в частности, программные средства аутентификации устройств на основе физически неклонируемых функций, а также особенности их реализации сами по себе являются коммерческой тайной. В связи с этим, даже поверхностный анализ и сравнение существующих аналогов не представляется возможным. Однако, целесообразным представляется создание доступного открытого аналога средства аутентификации, которое, в противовес проприетарным решениям компаний-производителей, могло бы служить удобной базой для дальнейшей развития и затачивания под конкретные нужды энтузиастами разработки цифровых устройств, а также в образовательных целях.


\subsection{Постановка задачи}
В результате выполнения дипломного проекта должно быть разработано программное средство аутентификации цифровых устройств, реализующее протокол взаимодействия между сервером аутентификации и цифровым устройством, включающим в себя некоторую реализацию PUF для однозначной его идентификации.
\begin{itemize}
\item Разрабатываемое ПО должно работать на операционных системах Linux и Windows.
\item Программное средство должно быть выполнено в виде клиент\hyphсерверного приложения.
\end{itemize}
