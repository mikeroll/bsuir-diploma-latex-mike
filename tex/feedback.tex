\thispagestyle{empty}

{\small
  \begin{center}
    \begin{minipage}{0.8\textwidth}
      \begin{center}
        {\normalsize ОТЗЫВ}\\[1em]
        о работе студента Булаша Михаила Валерьевича
        над дипломным проектом
        на тему:
        <<Программное средство распознавания уникальных неклонируемых идентификаторов цифровых устройств>>
      \end{center}
    \end{minipage}
  \end{center}

В рамках дипломного проектирования студент Булаш~М.\,В. провел анализ научно-технической литературы в области методов применения физически неклонируемых функций (ФНФ) для идентификации и проверки подлинности цифровых устройств, способов и протоколов их аутентификации в информационных системах или системах физической безопасности.

По результатам проведенного анализа была поставлена задача: реализовать комплекс программных средств для обеспечения инфраструктуры аутентификации цифровых устройств, опирающейся на использование физически неклонируемых функций в качестве уникальных идентификаторов этих устройств. Выполнение задачи потребовало от дипломника знаний современных технологий и языков программирования, понимания основ физической криптографии и навыков работы с защищёнными сетевыми протоколами для обеспечения достаточного уровня надёжности программного системы.

На основе спецификации требований было проведено проектирование ПС, в том числе разработка структуры ПС, его алгоритмов. Для подтверждения качества разработанного ПС было проведено тестирование. Разработано руководство пользователя. На достаточном уровне дополнительно проработан вопрос технико-экономического обоснования разработанного ПС.


Студент Булаш~М.\,В. проявил способность к самостоятельной работе. График выполнения дипломного проекта соблюден в полной мере.  Тщательно проведен обзор научно-технической литературы, показано умение работать с литературными источниками как на русском, так и на английском языке. В ходе дипломного проектирования Булаш~М.\,В. продемонстрировал глубокие теоретические знания современных языков программирования и, средств и подходов к разработке сложных программных систем, а также практические навыки в их применении.

Считаю, что Булаш~М.\,В. грамотно справился с поставленными задачами, а дипломный проект выплонен на высоком уровне. Студент Булаш~М.\,В. заслуживает присвоения квалификации инженер-программист по специальности «Программное обеспечение информационных технологий».

  \vfill
  \noindent
  \begin{minipage}{0.54\textwidth}
    \begin{flushleft}
      Руководитель дипломного проекта,
      проф. каф. информатики БГУИР \\
      д.т.н., доц.:\\
    \end{flushleft}
  \end{minipage}
  \begin{minipage}{0.44\textwidth}
    \begin{flushright}
      \underline{\hspace*{3cm}}~Иванюк~А.\,А.
    \end{flushright}
  \end{minipage}
}

\clearpage
