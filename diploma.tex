\input{tex/preamble}
\begin{document}

% Не пытаемся впихивать по максимуму - не получаем вылазящих за правый край слов
\sloppy

% Титульник
\begin{titlepage}
  \begin{center}
    Министерство образования Республики Беларусь\\[1em]
    Учреждение образования\\
    БЕЛОРУССКИЙ ГОСУДАРСТВЕННЫЙ УНИВЕРСИТЕТ \\
    ИНФОРМАТИКИ И РАДИОЭЛЕКТРОНИКИ\\[1em]

    \begin{minipage}{\textwidth}
      \begin{flushleft}
        \begin{tabular}{ l l }
          Факультет & Компьютерных систем и сетей\\
          Кафедра   & программного обеспечения информационных технологий
        \end{tabular}
      \end{flushleft}
    \end{minipage}\\[1em]

    \begin{flushright}
      \begin{minipage}{0.4\textwidth}
        \textit{К защите допустить:}\\[0.8em]
        Заведующая кафедрой ПОИТ\\[0.45em]
        \underline{\hspace*{2.8cm}} Н.\,В.~Лапицкая
      \end{minipage}\\[2.2em]
    \end{flushright}

    {ПОЯСНИТЕЛЬНАЯ ЗАПИСКА}\\
    {к дипломному проекту}\\
    {на тему}\\[1em]
    \textbf{\large ПРОГРАММНОЕ СРЕДСТВО РАСПОЗНАВАНИЯ УНИКАЛЬНЫХ НЕКЛОНИРУЕМЫХ ИДЕНТИФИКАТОРОВ ЦИФРОВЫХ УСТРОЙСТВ}\\[1em]


    {БГУИР ДП 1-40 01 01 03 013 ПЗ}\\[2em]

    \begin{tabular}{ p{0.65\textwidth}p{0.25\textwidth} }
      Студент & М.\,В.~Булаш \\
      Руководитель & А.\,А.~Иванюк \\
      Консультанты: &\\
      \hspace*{3ex}\emph{от кафедры ПОИТ} & А.\,А.~Иванюк \\
      \hspace*{3ex}\emph{по экономической части} & С.\,В.~Литвинович \\
      Нормоконтролёр & С.\,В.~Болтак\\
      & \\
      Рецензент &
    \end{tabular}

    \vfill
    {\normalsize Минск 2016}
  \end{center}
\end{titlepage}
 %

% Лист задания
\input{tex/taskpage} %

% Реферат
\sectioncentered*{Реферат}

% всего страниц + ведомость
\newcommand{\totpages}{\number\numexpr\getpagerefnumber{LastPage} + 1}

\begin{center}
    Пояснительная записка \totpages~с., \totfig{}~рис., \tottab{}~табл., \totref{}~источников.
    \\
    \MakeUppercase{цифровые устройства, физически неклонируемые функции, моделирование физической системы, машинное обучение, аутентификация, контроль доступа}
\end{center}

Объектом исследования дипломного проекта является возможность и необходимые условия практического использования физически неклонируемых функций (ФНФ) в системах физической криптографии, аутентификации и контроля доступа. В частности, рассматриваются способы аутентификации цифровых устройств, оборудованных ФНФ, в информационных системах.

Цель дипломного проектирования - разработка программной реализации протокола аутентификации цифровых устройств. Подробно рассмотрены и реализованы механизмы взаимодействия собственно устройства и сервера аутентификации на этапе регистрации устройства в информационной системе и на этапе проверки его подлинности.
Так как целью является именно реализация взаимодействия, устройство, оборудованное ФНФ решено выполнить в виде программной эмуляции в целях упрощения тестирования.

Разработка программной системы велась на языке Python, с использованием библиотек Twisted, Werkzeug для обеспечения асинхронного сетевого взаимодействия. В процессе работы использовались среда разработки Sublime Text 3 с расширениями SublimePythonIDE. Пояснительная записка оформлена с помощью технологии LaTeX и пакета утилит LatexTools. В ходе работы исследованы и применены алгоритмы математического моделирования физических систем и коррекции цифровых сигналов.

Результатом дипломного проекта стала легковесная программная система, реализующая протокол аутентификации физических устройств в информационной системе, и предоставляющая достаточный уровень защиты от несанкционированного доступа к секретным данным. Программный продукт позиционируется как основа для построения более сложной системы, например, системы контроля и управления доступом.

Разработанное приложение является экономически эффективным, оно полностью оправдывает средства, вложенные в его разработку.

\clearpage
 %

% Содержание
\clearpage \setcounter{page}{5}
\input{tex/toc} %

% Определения и сокращения
\sectioncentered*{Определения и сокращения}

В представленной пояснительной записке используются следующие определения и сокращения:

Аутентификация -- процедура проверки подлинности идентификатора, предъявленного сущностью для получения доступа к ресурсу.

IC -- Integrated Circuit (интегральная микросхема)

Process Variation -- вариация технического процесса, явление случайного отклонения значений физических параметров полупроводника от эталонных на этапе изготовления.

PUF -- Physically Unclonable Function (физически неклонируемая функция, ФНФ)

CRP -- Challenge-Response pair (пара запрос-ответ)

Уникальный неклонируемый идентификатор -- совокупность пар запрос-ответ, однозначно идентифицирующая конкретное устройство.

SRAM -- Static Random Access Memory (cтатическая память с произвольным доступом)

ПО -- программное обеспечение

ОС -- операционная система

ООП -- объектно-ориентированное программирование

ФП -- функциональное программирование

WSGI -- Web Server Gateway Interface, стандарт взаимодействия между программой, выполняющейся на стороне сервера, и самим веб-сервером.

API -- Application Programming Interface (интерфейс прикладного программирования, интерфейс программирования приложений)


\clearpage
 %

% Введение
\sectioncentered*{Введение}
\addcontentsline{toc}{section}{Введение}
\label{sec:intro}

Физическая криптография, основанная на структурной сложности физических систем, является одним из наиболее современных достижений в области криптографии и защиты информации. Данное направление в области защиты информации является значительным шагом вперед по сравнению с классической (алгебраической) криптографии, опирающейся только на математические методы и предоставляет качественно новые средства генерации и обработки информации, так как строится на основе эксплуатации шумоподобного поведения физических объектов и систем. Стойкость системы к статистическому криптоанализу в физической криптографии выведена на новый уровень, в основном благодаря истинной, плохо поддающейся математическому моделированию случайности значений параметров, извлекаемых из физической системы.

Главной сущностью физической криптографии является физически неклонируемая функция (ФНФ) -- физическая система, обладающая множеством компонент, параметры которых в процессе создания подобных физических систем принимают случайные значения. Главным свойством ФНФ является собственно неклонируемость -- безусловная невозможность создания точной копии строения или поведения этой системы.

В рамках данного дипломного проекта рассматриваются методы использования ФНФ в качестве уникальных идентификаторов физических сущностей, в частности -- электронных устройств. Логическим продолжением процесса идентификации сущности является аутентификация -- т.е. проверка подлинности предъявленного идентификатора. В представленном проекте изучены методы построения систем аутентификации таких идентификаторов, реализован легковесный протокол, решающий эту задачу, и программная система обмена секретной информацией по этому протоколу, которая может использоваться в качестве основы для построения систем информационной или физической безопасности.
 %

% Глава 1 Обзор предметной области
\section{Обзор предметной области}
\label{sec:domain:intro}

В данном разделе будет произведён обзор предметной области задачи, решаемой в рамках дипломного проекта - аутентификации цифровых устройств как механизма защиты интеллектуальной собственности; рассмотрены существующие методы аутентификации цифровых устройств с использованием физически неклонируемых функций, как теоретические, так и применяемые на практике; приведены преимущества и недостатки различных подходов, используемых при проектировании цифровых устройств и систем их аутентификации.
В частности, будут рассмотрены варианты построения аппаратных уникальных неклонируемых идентификаторов цифровых устройств, а так же механизмы и протоколы аутентификации устройств с использованием данных идентификаторов.


\subsection{Интеллектуальная собственность в полупроводниковой промышленности}
\label{sub:domain:ic_ip}
В современной полупроводниковой промышленности процесс проектирования устройств напрямую зависит от поставщиков отдельных модулей, составляющих эти устройства этих устройств. Эти модули рассчитаны на совокупное использование и заранее разрабатываются с учётом возможного применения в устройствах различной степени сложности и предусматривают относительно простое подключение друг к друг.  Данные блоки называются IP-ядрами или IP-блоками (IP-core, IP-block). При создании полупроводниковых устройств системные интеграторы покупают лицензию на IP-ядра для последующего использования в своих продуктах. Различают три основных вида ядер~\cite{counterfeit_ics}:
\begin{itemize}
\item Программные IP-блоки (soft blocks) - блоки, специфицированные  с помощью языка описанию аппаратуры или схожего по уровню абстракции языка, то они представляют собой процессно-инвариантные цифровые IP-блоки, которые могут быть использованы для синтеза на вентильном уровне. Программные блоки очень гибкие и могут быть легко перенесены из одной системы в другую.
\item Физические IP-блоки (hard blocks) - блоки, специфицированные на физическом уровне реализации, например Graphical Database System II (GDSII) для ASIC. Эти блоки обладают фиксированным расположением элементов и предсказуемыми временными характеристиками, энергопотреблением и занимаемой площадью. Их основной недостаток - жесткость дизайна (ограничены одним технологическим процессом) и слабая переносимость.
\item Схемотехнические IP-блоки (firm blocks) - блоки, специфицированные на схемотехническом уровне, без привязки к конкретной топологической реализации. Обычно предоставляются в форме готового списка соединений и более предсказуемы, чем программные блоки. Могут быть легко перенесены на различные технологические процессы и оптизированы для потребностей разрабатывающей стороны.
\end{itemize}

IP-блоки служат базовыми блоками для построения устройств и, таким образом, являются основными объектами интеллектуальной собственности применительно к разработке интегральных микросхем.Закономерным становится вопрос о защите IP-ядер как интеллектуальной собственности от несанкционированного использования, изменения, копирования. Способы защиты этих продуктов, а так же возможный вариант реализации такой системы защиты являются объектом исследования данного дипломного проекта.


\subsection{Физически неклонируемые функции}
\label{sub:domain:pufs}
Физическая криптография, в основе которой лежат физически неклонируемые функции, может найти применение не только в традиционном смысле защиты конфиденциальной информации, но также и в области защиты собственно электронных цифровых устройств от несанкционированного использования, изменения, клонирования. Идея применения физически неклонируемых функций (Phisycal Unclonable Function, PUF, или изначально Physical One-Way Function, POWF) впервые была предложена в работах Р. Паппу (R. Pappu) в 2001 году~\cite{rpappu_powf}, а современное толкование и определение сформулировал П. Туилс (P. Tuyls) в 2006 году~\cite{ptuyls_pufs}. Согласно этому определению, физически неклонируемой функцией называют набор характеристик физической (цифровой) системы, копирование и воспроизведение которого применительно к другим физическим системам невозможно. Цифровые системы, как правило, состоят из множества компонент, физические параметры которых формируются на стадии производства и принимают случайные в некотором диапазоне значения. Контролировать эти значения невозможно в силу хаотичности микро- (а часто и макро-)скопических параметров физической системы, в рамках которой происходит создание устройства. Подобное свойство получило название физической вариации технологического процесса~\cite{ivaniuk_pufs}. Таким образом, уникальность и невоспроизводимость (неклонируемость) цифровой системы обеспечивается наличием подобных случайных параметров. Более того, PUF могут быть реализованы таким образом, который не допускает симуляцию, моделирование или предсказание их поведения. Идея использования PUF в криптографических целях основана на извлечении и использовании этих случайных параметров.

\def\puf_crp {\left\{C_i, R_i\right\}}

PUF - это физическая система, которая при воздействии на неё (запросе) порождает уникальный, но непредсказуемый ответ. Специфический запрос C (Challenge) и соответствующий ему выходной сигнал R (Response) вместе образуют пару запрос-ответ (Challenge-Response Pair, CRP). PUF, в свою очередь, является функцией преобразования множества запросов $ C_i $ в множество ответов $ R_i $:
\begin{equation}
  \label{eq:domain:pufs:puf_main}
  R_i = PUF(C_i)
\end{equation}

По определению У. Рурмаира (U. Ruhrmair), PUF представляют собой сложные неуправляемые физические системы со сверхбольшим объёмом структурной информации (Super High Information Content, SHIC). PUF должны удовлетворять следующим свойствам:
\begin{enumerate}
  \item Физическое копирование системы с сохранением её свойств в виде пар $ \puf_crp $ должно быть практически невозможно.
  \item Информация в виде ответов $ R_i $ на запросы $ C_i $ может быть извлечена из системы многократно и с высокой степенью надёжности.
  \item Множество возможных запросов $ C_i $ должно быть достаточно велико, чтобы получение всех пар $ \puf_crp $ не представлялось возможным за разумное время и количество ресурсов.
  \item Пары $ \puf_crp $ должны быть независимы друг от друга в том смысле, что по известной паре $ \puf_crp $ невозможно получить, смоделировать или предсказать значение любой другой пары или множества пар.
\end{enumerate}


% Виды ФНФ

\subsection{Виды PUF по физической реализации. }
\label{sub:domain:puf_physical_types}


\subsubsection{PUF на оптических элементах. }
\label{sub:domain:puf_physical_types:optical}
Любые физически неклонируемые функции, в которых процесс реакции системы в основе своей имеет какое-либо оптическое явление, называются оптическими.
Как правило, используется объект с неоднородной прозрачностью (подобно светоотражающим пузырькам из примера выше). Именно благодаря рассеивающему действию вещества и вкраплений объект не может быть исследован и смоделирован математически. Оптическое сканирование не может проникнуть вглубь объекта.


\subsubsection{PUF на интегральных микросхемах. }
\label{sub:domain:puf_physical_types:ic}
Логично предположить, что аналогично пузырькам в оптической системе, в иных средах тоже можно получить неклонируемость. И логично рассмотреть среду, и так широко используемую в цифровых устройствах - электрические проводники и диэлектрики.
Любое современное технологичное устройство содержит в себе внушительное количество интегральных микросхем. На них и строится <<электрическая>> PUF. Несмотря на то, что микросхемы изготавливаются по одному и тому же технологическому процессу, каждая из них достаточно уникальна для корректной работы PUF. Это может быть использовано в системах ИБ как уникальный идентификационный признак устройства.
ИМС покрывается слоем защитного вещества со вкраплениями диэлектрика. Эти вкрапления имеют случайные размер и форму. Под этот слой подводятся электроды-датчики. В совокупности с защитным слоем каждый такой электрод обретает свойства конденсатора случайной (зависящей от вкраплений диэлектрика в защитный слой) ёмкости. Очевидно, что случайность ёмкости каждого конденсатора достигается при размере частиц, сравнимом с размером между электродами.


\subsubsection{PUF на полевых транзисторах. }
\label{sub:domain:puf_physical_types:transistors}
В основе таких PUF лежит особенность полевых транзисторов задерживать сигнал, проходящий по нему на непредсказуемое время, зависящее от физических свойств материала транзистора (именно из-за материала такие PUF ещё называют кремниевыми). Физическая система PUF будет состоять из набора пар транзисторов и триггеров-арбитров (см. ниже PUF с арбитром). Триггер будет давать на выход 0 или 1 в зависимости от того, сигнал с какого из пары транзисторов пришёл к нему раньше. Подавая на вход этому набору некоторый сигнал-запрос, на выходе исследователь получит набор значений триггеров - реакцию системы на запрос, уникальную для данного устройства, в состав которого включаются упомянутые транзисторы. Неклонируемость строится на физической неидеальности процесса производства. Изучение функции для последующего математического прогноза реакции на запрос возможно только полным перебором входных сигналов, что является достаточно вычислительно сложной задачей.


\subsubsection{PUF на магнитных элементах. }
\label{sub:domain:puf_physical_types:magnetic}
Практическое применение - уникальный идентификатор магнитной полосы банковской карты. Частицы феррита бария, содержащиеся в пасте-основе магнитной полосы, также имеют случайный размер и форму. Логично сделать вывод, что на случайности их распределения можно построить PUF. Неклонируемость снова зиждется на физической неидеальности процесса производства. Неточности и погрешности не дадут повторить рисунок частиц феррита бария в точности. А математическое моделирование не препятствует выполнению данной PUF её задачи.


\subsection{Виды PUF по принципу работы. }
\label{sub:domain:puf_types}


\subsubsection{PUF типа арбитр (Arbiter PUF). }
\label{sub:domain:puf_types:arbiter}
Физически неклонируемые функции по типу арбитра - разновидность физически неклонируемых функций на основе задержек. Идея состоит в том, чтобы привнести состояние гонки между двумя путями микросхемы. Оба пути заканчиваются элементом-арбитром, который определяет какой из путей был быстрее и выдает соответствующее бинарное значение.


\subsubsection{PUF на базе кольцевого генератора (RO-PUF). }
\label{sub:domain:puf_types:ring_oscillator}
Физически неклонируемые функции на базе кольцевых генераторов также используют неконтролируемые изменения задержек процессов цифровых компонентов в качестве источника случайности. Когда эти компоненты образуют кольцевой генератор, частоты его выходных сигналов различаются, что и используется при формировании бинарного ответа.


\subsubsection{PUF на основе сбоев (Failure PUF). }
Этот тип физически неклонируемых функций основываются на сбоях в поведении комбинаторных логических схем. В идеале у комбинаторной схемы нет внутреннего состояния, что означает, что стационарный выход полностью определяется его входными сигналами. Однако, когда логическое значение на входе изменяется, для достижения стационарного значения на выходе требуется некоторое время. Появление сбоя определяется различиями в задержках различных логических цепей от входов к выходному сигналу. Так как задержки определенных образцов комбинаторных схем вызваны случайными изменениями процесса, появление, количество и форма сбоев выходных сигналов также будут случайными и характерными для определенных образцов схемы. Поэтому оценка поведения сбоев таких схем может быть использована для ответа физически неклонируемой функции.


\subsubsection{PUF на базе SRAM (SRAM PUF). }
\label{sub:domain:puf_types:sram}
Принцип работы физически неклонируемых функций на базе статического оперативного запоминающего устройства (СОЗУ) основан на случайности состояния части ячеек СОЗУ при включении.


\subsubsection{PUF типа бабочка (Butterfly PUF). }
\label{sub:domain:puf_types:arbiter}
Физически неклонируемые функции по типу бабочки имитирует работу ячейки СОЗУ, формируя перекрестные обратные связи. Получается бистабильная схема. Схема принудительно переводится в неустойчивое состояние, после чего схема переходит в одно из двух стабильных состояний, которое зависит от случайной разности задержек в паре линий обратной связи и линии входного сигнала.

Стоит заметить, что в данном списке перечислены лишь базовые варианты реализации PUF. На их основе и на основе комбинаций этих типов может быть построено огромное множество различных сложных PUF. ~\cite{cryptowiki_pufs, rmaes_pufs}


\subsection{Использование PUF для идентификации и аутентификации цифровых устройств}
\label{sub:domain:puf_auth}
Благодаря тому, что физически неклонируемые функции уникальны и сложно воспроизводимы, они могут быть использованы для целей аутентификации. Сценарий аутентификации представлен централизованным проверяющим и объектами (устройствами), чья подлинность проверяется.

Существует два основных подхода к осуществлению системы аутентификации, основанной на физически неклонируемых функциях. Первый состоит в разработке схемы аутентификации, которая напрямую применяет уникальность и непредсказуемость поведения пары запроса-ответа отдельного объекта. Такая схема состоит из двух фаз (регистрации и подтверждения):

Сначала каждый объект проходит регистрацию у проверяющего. Во время этой фазы проверяющий записывает идентификатор каждого объекта и собирает значительное подмножество пар запрос-ответ каждого объекта (устройства) для случайно сгенерированных запросов. Собранные пары запрос-ответ хранятся в базе данных проверяющего.
Во время фазы подтверждения, объект (устройство) посылает проверяющему свой идентификатор. Проверяющий находит его в базе данных, выбирает оттуда случайную пару запрос-ответ, которая соответствует полученному идентификатору. Запрос посылается объекту, объект вычисляет физически неклонируемую функцию и посылает ответ. Проверяющий сравнивает, насколько ответ близок к значению из базы данных, то есть что ответы различаются не более чем на заранее установленное значение. Если проверка прошла успешно, объект аутентифицирован, иначе - аутентификация отклонена. Использованная пара запрос-ответ удаляется из базы данных.
Корректность данной схемы аутентификации обеспечивается тем фактом, что ответы физически неклонируемых функций воспроизводимы самим объектом в течение долгого времени.

Второй подход состоит в получении стойкого и безопасного криптографического ключа из ответа физически неклонируемой функции и использовании этого ключа в каком-либо из существующих криптографических протоколов аутентификации.

При внедрении физически неклонируемой функции в кредитные или смарт-карты, RFID-метки, ценные бумаги и т. д., эти объекты становятся неклонируемыми, а их идентичности проверяемыми с помощью одного из методов, описанных выше.


\subsection{Обзор существующих аналогов}
Подавляющее большинство средств защиты интеллектуальной собственности применительно к цифровым устройствам, в частности, программные средства аутентификации устройств на основе физически неклонируемых функций, а также особенности их реализации сами по себе являются коммерческой тайной. В связи с этим, даже поверхностный анализ и сравнение существующих аналогов не представляется возможным. Однако, целесообразным представляется создание доступного открытого аналога средства аутентификации, которое, в противовес проприетарным решениям компаний-производителей, могло бы служить удобной базой для дальнейшей развития и затачивания под конкретные нужды энтузиастами разработки цифровых устройств, а также в образовательных целях.


\subsection{Постановка задачи}
В результате выполнения дипломного проекта должно быть разработано программное средство аутентификации цифровых устройств, реализующее протокол взаимодействия между сервером аутентификации и цифровым устройством, включающим в себя некоторую реализацию PUF для однозначной его идентификации.
\begin{itemize}
\item Разрабатываемое ПО должно работать на операционных системах Linux и Windows.
\item Программное средство должно быть выполнено в виде клиент\hyphсерверного приложения.
\end{itemize}
 %

% Глава 2 Используемые технологии
\section{Используемые технологии}
\label{sec:techs:intro}

Выбор технологий является важным предварительным этапом разработки сложных информационных систем.
Платформа и язык программирования, на котором будет реализована система, заслуживает большого внимания, так как исследования показали, что выбор языка программирования влияет на производительность труда программистов и качество создаваемого ими кода~\cite[c.~59]{mcconnell_2005}.

Ниже перечислены некоторые факторы, повлиявшие на выбор технологий:
\begin{itemize}
\item Разрабатываемое ПО должно работать на операционных системах Linux и Windows.
\item Программное средство должно быть выполнено в виде клиент-серверного приложения.
\item Дальнейшей поддержкой проекта, возможно, будут заниматься разработчики, не принимавшие участие в выпуске первой версии.
\item Имеющийся разработчик имеет опыт работы с объектно"=ориентированными и с функциональными языками программирования.
\end{itemize}

Основываясь на приведенных факторах, целесообразно выбрать в качестве платформы разработки язык Python. Приняв во внимание необходимость обеспечения доступности дальнейшей поддержки ПО, возможно, другой командой программистов, целесообразно не использовать малоизвестные и сложные языки программирования.

Для реализации поставленной задачи существует необходимость в написании серверного приложения. Эту задачу можно облегчить путём использования набора прикладных библотек (фреймворка) Flask, предназначенного для быстрого прототипирования и реализации веб-приложений различной степени сложности.

Далее проводится характеристика используемых технологий и языков программирования более подробно.



\subsection{Язык программирования Python}
\label{sub:techs:python}
Python — высокоуровневый язык программирования общего назначения, ориентированный на повышение производительности разработчика и читаемости кода. Синтаксис ядра Python минималистичен. В то же время стандартная библиотека включает большой объём полезных функций.

Python поддерживает несколько парадигм программирования, в том числе структурное, объектно-ориентированное, функциональное, императивное и аспектно-ориентированное. Основные архитектурные черты — динамическая типизация, автоматическое управление памятью, полная интроспекция, механизм обработки исключений, поддержка многопоточных вычислений и удобные высокоуровневые структуры данных. Код в Python организовывается в функции и классы, которые могут объединяться в модули (они в свою очередь могут быть объединены в пакеты).

Эталонной реализацией Python является интерпретатор CPython, поддерживающий большинство активно используемых платформ. Он распространяется под свободной лицензией Python Software Foundation License, позволяющей использовать его без ограничений в любых приложениях, включая проприетарные. Есть реализации интерпретаторов для JVM (с возможностью компиляции), MSIL (с возможностью компиляции), LLVM и других. Проект PyPy предлагает реализацию Python на самом Python, что уменьшает затраты на изменения языка и постановку экспериментов над новыми возможностями.

Python — активно развивающийся язык программирования, новые версии (с добавлением/изменением языковых свойств) выходят примерно раз в два с половиной года. Вследствие этого и некоторых других причин на Python отсутствуют стандарт ANSI, ISO или другие официальные стандарты, их роль выполняет CPython.

Отличительные особенности языка Python:
\begin{itemize}
  \item Простой. Python – простой и минималистичный язык. Чтение хорошей программы на Python очень напоминает чтение английского текста. Такая псевдо-кодовая природа Python является одной из его самых сильных сторон. Она позволяет сосредоточиться на решении задачи, а не на самом языке.
  \item Свободный и открытый. Python – это пример свободного и открытого программного обеспечения – FLOSS (Free/Libré and Open Source Software). Пользователь имеет право свободно распространять копии этого программного обеспечения, читать его исходные тексты, вносить изменения, а также использовать его части в своих программах. В основе свободного ПО лежит идея сообщества, которое делится своими знаниями. Это одна из причин, по которым Python так хорош: он был создан и постоянно улучшается сообществом, которое хочет сделать его лучше.
  \item Язык высокого уровня. При написании программы на Python разработчику не нужно отвлекаться на такие низкоуровневые детали, как управление памятью и т.п.
  \item Портируемый. Благодаря своей открытой природе, Python был портирован на множество платформ. Программы на Python могут запускаться на любой из этих платформ без каких-либо изменений (если программа не использует системно-зависимые функции). Python можно использовать в GNU/Linux, Windows, FreeBSD, Macintosh, Solaris, OS/2, Amiga, AROS, AS/400, BeOS, OS/390, z/OS, Palm OS, QNX, VMS, Psion, Acorn RISC OS, VxWorks, PlayStation, Sharp Zaurus, Windows CE и множестве других платформ.
  \item Интерпретируемый. Python не требует компиляции в бинарный код. Программа выполняется из исходного текста. Python сам преобразует этот исходный текст в некоторую промежуточную форму, называемую байткодом, а затем переводит его на машинный язык и запускает. Всё это заметно облегчает использование Python, поскольку нет необходимости заботиться о компиляции программы, подключении и загрузке нужных библиотек и т.д. Вместе с тем, это делает программы на Python намного более переносимыми.
  \item Объектно-ориентированный. Python поддерживает как процедурно-ориентированное, так и объектно-ориентированное программирование. Python предоставляет простые, но мощные средства для ООП, особенно в сравнении с такими большими языками программирования, как C++ или Java.
  \item Расширяемый. Если необходимо добиться очень высокой производительности некоторой части программы или использовать некоторые другие возможности более низкоуровневых языков, можно разработать эту часть программы на C или C++, а затем вызывать её из программы на Python.
  \item Встраиваемый. Python можно встраивать в программы на C/C++, чтобы предоставлять возможности написания сценариев их пользователям.
  \item Обширные библиотеки. Стандартная библиотека Python просто огромна. Она может помочь в решении самыхразнообразных задач, связанных с использованием регулярных выражений, генериро-ванием документации, проверкой блоков кода, распараллеливанием процессов, база-ми данных, веб-браузерами, CGI, FTP, электронной почтой, XML, XML-RPC, HTML, WAV файлами, криптографией, GUI и другими системно-зависимыми вещами. Всё это доступно абсолютно везде, где установлен Python. В этом заключается философия Python <<Всё включено>>.Кроме стандартной библиотеки, существует множество других высококачественных биб-лиотек, доступных в каталоге пакетов.
\end{itemize}

\subsubsection{Объектно-ориентированное программирование. }
Дизайн языка Python построен вокруг объектно-ориентированной модели программирования. Реализация ООП в Python является элегантной, мощной и хорошо продуманной, но вместе с тем достаточно специфической по сравнению с другими объектно-ориентированными языками. Особенности~\cite{wiki_python, byte_of_python}:
\begin{itemize}
\item Классы являются одновременно объектами со всеми ниже приведёнными возможностями.
\item Наследование, в том числе множественное.
\item Полиморфизм (все функции виртуальные).
\item Инкапсуляция (два уровня — общедоступные и скрытые методы и поля). Особенность — скрытые члены доступны для использования и помечены как скрытые лишь особыми именами.
\item Специальные методы, управляющие жизненным циклом объекта: конструкторы, деструкторы, распределители памяти.
\item Перегрузка операторов (всех, кроме is, '.', '=' и символьных логических).
\item Свойства (имитация поля с помощью функций).
\item Управление доступом к полям (эмуляция полей и методов, частичный доступ, и т. п.).
\item Методы для управления наиболее распространёнными операциями (истинностное значение, len(), глубокое копирование, сериализация, итерация по объекту, …)
\item Метапрограммирование (управление созданием классов, триггеры на создание классов, и др.)
\item Классовые и статические методы, классовые поля.
\end{itemize}

\subsubsection{Функциональное программирование. }
Python поддерживает парадигму функционального программирования, в частности:
\begin{itemize}
\item Функция является объектом;
\item Функции высших порядков;
\item Рекурсия;
\item Развитая обработка списков (списочные выражения, операции над последовательностями, итераторы);
\item Аналог замыканий;
\item Частичное применение функции;
\end{itemize}

\subsubsection{Интроспекция. }
Python поддерживает полную интроспекцию времени исполнения. Это означает, что для любого объекта можно получить всю информацию о его внутренней структуре.
Применение интроспекции является важной частью того, что называют pythonic style, и широко применяется в библиотеках и фреймворках Python, таких как PyRO, PLY, Cherry, Django и др., значительно экономя время использующего их программиста.

\subsubsection{Итераторы. }
В программах на Python широко используются итераторы. Цикл for может работать как с последовательностью, так и с итератором. Все коллекции, как правило, предоставляют итератор. Объекты определённого пользователем класса тоже могут быть итераторами. Подробнее об итераторах можно узнать в разделе о функциональном программировании. Модуль itertools стандартной библиотеки содержит много полезных функций для работы с итераторами.

\subsubsection{Генераторы. }
Одной из интересных возможностей языка являются генераторы — функции, сохраняющие внутреннее состояние: значения локальных переменных и текущую инструкцию. Генераторы могут использоваться как итераторы для структур данных и для ленивых вычислений.
При вызове генератора функция немедленно возвращает объект-итератор, который хранит текущую точку исполнения и состояние локальных переменных функции. При запросе следующего значения (посредством метода next(), неявно вызываемого в цикле for) генератор продолжает исполнение функции от предыдущей точки останова до следующего оператора yield или return.

\subsection{Язык описания аппаратуры интегральных схем VHDL}
\label{sub:techs:vhdl}
Возрастающая алгоритмическая сложность аппаратно реализованных устройств приводит к тому, что, как проблемы разработки, описания и применения аппаратуры (hardware), так и подходы к их решению, становятся подобны проблемам и методам решения для современных программных систем (software). Перспективное направление решения этих проблем — применение алгоритмического подхода, создание алгоритмического языка для описания аппаратуры, программирования и структуры, функционирования аппаратных средств обработки информации. Наиболее распространенным языком этого класса, специфицированным международными стандартами, является язык VHDL, который разработан в рамках американского проекта создания нового поколения высокоскоростной элементной базы (Very High Speed Integrated Circuits — VHSIC). Аббревиатура VHDL расшифровывается как VHSIC Hardware Description Language. Расширение языка VHDL — язык VHDL-AMS (Very-High-Speed IС Hardware Description Language — Analog and Mixed Signal) включает также возможности моделирования систем, содержащих и цифровую, и аналоговую части.~\cite{suvorova_vhdl, ivchenko_vhdl}

Язык VHDL предназначен для решения комплекса задач в ходе проектирования и применения цифровых систем, их аппаратных средств, в том числе:
\begin{itemize}
  \item  Описания структуры системы, декомпозиции системы на подсистемы,
спецификации связей и взаимодействия подсистем.
  \item  Спецификации функционирования системы, узлов, блоков, реализуемых
функций. Спецификация дается в алгоритмической форме, с использова-
использованием привычных современному специалисту программных конструкций
алгоритмического языка, включающих в себя спецификацию временного
поведения сигналов и блоков.
  \item  Моделирования системы и ее работы на основе четкой спецификации
структуры системы, а также функционирования ее компонентов.
  \item  Синтеза схемотехнической реализации системы, автоматической генера-
генерации детальной структуры на основе строгой спецификации системы на
языке VHDL — спецификации на более абстрактном уровне.
\end{itemize}

Первоначально язык предназначался для моделирования, но позднее из него было выделено синтезируемое подмножество. Написание модели на синтезируемом подмножестве позволяет автоматический синтез схемы функционально эквивалентной исходной модели. Средствами языка VHDL возможно проектирование на различных уровнях абстракции (поведенческом или алгоритмическом, регистровых передач, структурном), в соответствии с техническим заданием и предпочтениями разработчика. Заложена возможность иерархического проектирования, максимально реализующая себя в экстремально больших проектах с участием большой группы разработчиков. Представляется возможным выделить следующие три составные части языка: алгоритмическую — основанную на языках Ada и Pascal и придающую языку VHDL свойства языков программирования; проблемно ориентированную — в сущности и обращающую VHDL в язык описания аппаратуры; и объектно-ориентированную, интенсивно развиваемую в последнее время.

VHDL является формальной записью, предназначенной для описания функций и логической организации цифровой системы. Функция системы определяется как преобразование значений на входах в значения на выходах. Причем время в этом преобразовании задается явно. Организация системы задается перечнем связанных компонентов.

\subsubsection{Структура VHDL-проекта. }

Объект проекта (entity) представляет собой описание компонента проекта, имеющего четко заданные входы и выходы и выполняющей четко определенную функцию. Объект проекта может представлять всю проектируемую систему, некоторую подсистему, устройство, узел, стойку, плату, кристалл, макроячейку, логический элемент и т.п.

В описании объекта проекта можно использовать компоненты, которые, в свою очередь, могут быть описаны как самостоятельные объекты проекта более низкого уровня. Таким образом, каждый компонент объекта проекта может быть связан с объектом проекта более низкого уровня. В результате такой декомпозиции объекта проекта пользователь строит иерархию объектов проекта, представляющих весь проект в целом и состоящую из нескольких уровней абстракций. Такая совокупность объектов проекта называется иерархией проекта (design hierarchy).Каждый объект проекта состоит, как минимум, из двух различных типов описаний: описания интерфейса и одного или более архитектурных тел.Интерфейс описывается в объявлении объекта проекта  (entity declaration)  и определяет только входы и выходы объекта проекта. Для описания поведения объекта или его структуры служит архитектурное тело (architecture body). Чтобы задать, какие объекты проекта использованы для создания полного проекта, используется объявление конфигурации (configuration declaration).

В языке VHDL  предусмотрен механизм пакетов для часто используемых описаний, констант, типов, сигналов. Эти описания помещаются в объявлении пакета (package declaration).Если пользователь использует нестандартные операции или функции, их интерфейсы описываются в объявлении пакета, а тела содержатся в теле пакета (package body).

Таким образом, при описании цифровой системы на языке VHDL,  пользователь может использовать пять различных типов описаний: объявление объекта проекта, архитектурное тело, объявление конфигурации, объявление пакета и тело пакета. Каждое из описаний является самостоятельной конструкцией языка  VHDL, может быть независимо проанализировано анализатором и поэтому получило название <<Модуль проекта>> (designunit).

Модули проекта, в свою очередь, можно разбить на две категории: первичные и вторичные. К первичным модулям относятся различного типа объявления. Ко вторичным  -  отдельно анализируемые тела первичных модулей. Один или несколько модулей проекта могут быть помещены в один файл, называемый файлом проекта (design file). Каждый проанализированный модуль проекта помещается в библиотеку проекта (design ibrary) и становится библиотечным модулем (library unit). Данная реализация позволяет создать любое число библиотек проекта. Каждая библиотека проекта в языке  VHDL имеет логическое имя (идентификатор). Фактическое имя файла, содержащего эту библиотеку, может совпадать или не совпадать с логическим именем библиотеки проекта. Для ассоциации логического имени библиотеки с соответствующим ей фактическим именем предусмотрен специальный механизм установки внешних ссылок.

По отношению к сеансу работы  VHDL существует два класса библиотек проекта: рабочие библиотеки и библиотеки ресурсов.Рабочая библиотека  -  это библиотека, с которой в данном сеансе работает пользователь и в которую помещается библиотечный модуль, полученный в результате анализа модуля проекта. Библиотека ресурсов  -  это библиотека, содержащая библиотечные модули, ссылка на которые имеется в анализируемом модуле проекта. В каждый конкретный момент пользователь работает с одной рабочей библиотекой и произвольным числом библиотек ресурсов.

Возможность создания и использования многих библиотек ресурсов позволяет пользователю классифицировать библиотечные модули по различным признакам. Например, водной библиотеке хранить описания микросхем одной серии, в другой  -  описания микросхем другой серии и т.д.    Или водной библиотеке хранить описания микросхемс одним типом задержки,  в другой  -  описания микросхем с другим типом задержки и т.д.
 %

% Глава 3 Проектирование архитектуры программного средства
\section{Архитектура и модули системы}

Разработанное программное обеспечение является сложным программным комплексом, состоящим из различных модулей:
\begin{itemize}
    \item модуль построения компактной модели PUF;
    \item модуль симуляции PUF;
    \item модуль контроля доступа;
    \item модуль проверки подлинности.
\end{itemize}

Вышеописанные модули опираются на следующие группы классов, разработанные в рамках дипломного проекта:
\begin{itemize}
    \item Библиотека протокола аутентификации. Написана на языке Python. Содержит методы для проведения основных операций, используемых в проктоколе аутентификации.
    \item Библиотека машинного обучения (для построения модели PUF). Написана на языке программирования Haskell и содержит реализацию алгоритма наивного байесовского классификатора (Naive Bayes Classifier), о котором будет рассказано далее.
    \item Набор классов для работы с PUF. Написана на языке Python. Содержит классы для манипуляции с объектами типа Challenge и Response, которые построены на базе битового массива (bitarray). Реализует реализации алгоритмов генерации, сравнения, сложения и других операций над битовыми массивами, необходимых для обработки сингалов PUF.
    \item Слой взаимодействия с устройствами на языке Python. Является легковесным контрактом, предоставляющим единый интерфейс обращения к подключенным устройствам.
\end{itemize}

Общий принцип взаимодействия компонентов и пример сценария использования программной системы приведен на рисунке~\ref{fig:architecture:flow}. В следующих подразделах более подробно описано назначение и устройство каждого из логических модулей ПС.

\afterpage{
  \begin{landscape}
  \thispagestyle{lscape}
  \begin{figure}[t!]
  \centering
    \includegraphics[scale=0.40]{flow.png}
    \caption{ Схема работы программной системы }
    \label{fig:architecture:flow}
  \end{figure}
  \end{landscape}
}


\subsection{Модуль построения компактной модели PUF}
Главная особенность реализуемого протокола аутентификации -- использование компактной модели PUF в качестве эталонного показателя подлинности. Поэтому, закономерно рассмотреть особенности построения этой модели.

Для построения компактной модели ФНФ был выбран наивный байесовский классификатор -- широко используемый метод машинного обучения~\cite{manning_ir}, который при своей относительной простоте реализации, позволяет добиться очень неплохих результатов классификации. На выбор повлияло также то, что задача обучения модели ФНФ сводится к задаче классификации множества входных битовых строк $C$ на выходные классы -- <<0>> и <<1>>.

Наивный баейсовский классификатор – семейство простых вероятностных классфикаторов, которые основываются на теореме Байеса. Классификатор использует решающее правило MAP (maximum a posteriori), которое ставит в соответствие объекту наиболее вероятную для него метку и описывается формулой~\cite{mitchell_ml}:
\begin{equation}
  \label{eq:architecture:bayes}
  y = \underset{c\,\in\,C}{argmax}~P(C) \prod\limits_{i=1}^{n} P(x_i|C)
\end{equation}
\begin{explanation}
где & $ P(C) $ & априорная вероятность принадлежности объекта к классу $C$; \\
    & $ P(x_i|C) $ & правдоподобие принадлежности объекта к классу $C$, исходя из значения аттрибута
$x_i$; \\
    & $ x_i $ & атрибут объекта.
\end{explanation}

При работе с непрерывными аттрибутами используется предположение, что аттрибуты выбираются из независимых непрерывных нормальных распределений. Таким образом, задача обучения классификатора заключается в нахождении параметров распределения -- математического ожидания и дисперсии -- для каждого из аттрибутов, что позволит вычислять $P(x_i|C)$.

Известно, что величины случайных отклонений, связанные с вариацией технологического процесса, подчняются нормальному распределению. Поэтому вполне очевидным является использование байесовского классификатора на основе нормального закона распределения:
\begin{equation}
  \label{eq:architecture:gaussian_bayes}
  P(x_i = \nu|C) = \frac{1}{\sqrt{2 \pi {\sigma}_{ic}^2}} \, exp(-\frac{\nu - \overline{x_{ic}}^2}{2\sigma_{ic}^2})
\end{equation}
\begin{explanation}
где & $x_{ic}$ & среднее значение атрибута, рассчитанное для объектов, принадлежащих
классу $C$; \\
    & $ \sigma_{ic}^2 $ & выборочная дисперсия значания атрибута объектов из класс $C$.
\end{explanation}

Выборочная дисперсия значения атрибута определяется формулой:
\begin{equation}
  \label{eq:architecture:dispersion}
  \sigma_{ic}^2 = \frac{1}{n - 1} \sum\limits_{x \in C} (x_i - \overline{x_{ic}}^2)
\end{equation}

В нашем случае, когда классифицируемыми объектами являются битовые массивы, удобно представить их виде векторов $s$ значених их битов: $s_i \in \{0, 1\}$. Тогда каждый бит массива будет являться атрибутом классифицируемого объекта. В дополнение к этому, благодаря ограниченному набору значений атрибутов ($\{0, 1\}$), вычисления можно значительно оптимизировать.

Модуль построения компактной модели PUF реализован на языке Haskell. На рисунке TODO представлена схема алгоритма классификатора. Листинги \ref{lst:architecture:naive_bayes_hs} и \ref{lst:architecture:statistics_hs} содержит фрагменты кода программного модуля, непосредственно относящиеся к решению задачи классификации. Модуль классификатора поддерживает работу в параллельном режиме на многопроцессорных устройствах, что значительно сокращает время обучения модели.


\lstinputlisting[
    style=commonstyle,
    caption={Наивный байесовский классификатор},
    label=lst:architecture:naive_bayes_hs
]{src/NaiveBayes.hs}

\lstinputlisting[
    style=commonstyle,
    caption={Статистические функции, используемые в классификаторе},
    label=lst:architecture:statistics_hs
]{src/Statistics.hs}

\begin{figure}[!h]
    \centering
    \includegraphics[width=0.95\textwidth]{bayes.png}
    \caption{Схема алгоритма обучения модели методом наивного байесовского классификатора}
    \label{fig:architecture:bayes}
\end{figure}

\subsection{Модуль симуляции PUF}
В целях упрощения разработки, тестирования и демонстрации функциональности ПС было принято решение использовать программную симуляцию физически неклонируемой функции вместо аппаратной реализации. Более того, симуляция PUF может применяться также и конечным пользователем, к примеру, для моделирования PUF со статистическими параметрами, приближенными к таковым у реальной PUF и определения оптимальных параметров функционирования протокола. Таким образом, симуляция PUF из удобного инструмента на этапе разработки переросла в одну из функций программного средства.

Принцип генераций случайных величин, которые будут использованы в качестве физических параметров виртуальной PUF, основан на том факте, что естественные отклонения физических параметров транзисторов (длина, ширина, толщина оксидной плёнки) от заданных значений на этапе их изготовления моделируются распределением, близким к нормальному, так как являются следствиями недетерминированных физических процессов~\cite{gauss_wiki,process_variation}.

\begin{figure}[!h]
    \centering
    \includegraphics[width=0.85\textwidth]{gauss.png}
    \caption{График функции плотности вероятности для нормального распределения}
    \label{fig:architecture:normal_pd}
\end{figure}

При симуляции физически неклонируемой функции важно учитывать некоторые статистические характеристики, обусловленные нормальным законом распределения значений величин отклонений физических параметров. В частности, из данного факта следует следующее:
\begin{itemize}
  \item Вероятность получения на выходе единицы равна вероятности получения на выходе нуля для всего множества векторов входных сигналов, т.е. $P(r = 0) = P(r = 1) = 0,5 $. Следовательно, в идеальном случае ровно половина возможных векторов входных сигналов соответствует единице на выходе, и ровно половина -- нулю.
  \item Ответы на похожие запросы имеют большую вероятность совпадения. Под похожими понимаются запросы с минимальным расстояниями между ними, рассчитанным по формуле Хэмминга. Иными словами, вероятность того, что ответы $r_0$ и $r_1$ на запросы $C_0$ и $C_1$ будут иметь разные значения, является монотонно возрастающей функцией от расстояния Хэмминга между запросами, т.е. $P(r_0 \neq r_1) = f(HD(C_0, C_1))$. На рисунке ФФФ показана данная зависимость на основании экспериментальных данных.
\end{itemize}

\begin{figure}[!h]
    \centering
    \includegraphics[width=0.75\textwidth]{avalanche.png}
    \caption{График влияния различия входных сигналов на вероятность различия выходных}
    \label{fig:architecture:avalanche}
\end{figure}

При идеальном распределении величин отклонений по нормальному закону выполняется строгий лавинный критерий~\cite{rfid0_puf}, который гласит, что при изменении одного бита аргумента функции выходной бит меняет значение на противоположное с вероятностью 1/2.

Не стоит также забывать, что случайная природа ФНФ не ограничивается лишь вариациями процесса изготовления устройства. Свою случайную компоненту также вносит так называемый <<шум>> -- физические условия окружающей среды, влияющие на параметры полупроводникового устройства. Наиболее остро ощущается влияние высоких температур, при которых функционирует устройство.

Таким образом, для симуляции PUF необходимо учитывать задержки распространения сигналов, привнесенные на этапе производства, а также влияния внешней среды, корректирующие значения этих параметров во время функционирования устройства. Код, используемый для симуляции PUF в программном средстве, учитывает оба эти компонента.

\lstinputlisting[
    style=commonstyle,
    caption={Функиции, используемые для симуляции ФНФ},
    label=lst:architecture:puf_sim
]{src/ropuf.py}


\subsection{Слой взаимодействия с устройствами}
Данная часть программного средства представляет собой абстракцию над любыми типами устройств с PUF (как аппаратными, так и программно симулированными) и предоставляет единый интерфейс взаимодействия с ними. Таким образом, благодаря этому слою, все части программного средства могут единообразно обращаться к любому подключённому устройству с целью регистрации или аутентификации вне зависимости от реализации этого устройства. Для эффективного взаимодействия в рамках этих целей достаточно лишь иметь доступ к собственно функции PUF для обмена сигналами и к некоторым её свойствам, в частности -- ожидаемой длине входного сигнала. С точки зрения программного кода этот слой реализован в виде абстрактного класса device.Device, исходный код которого представлен в листинге \ref{lst:architecture:device}.

\lstinputlisting[
    style=commonstyle,
    caption={Слой взаимодействия с устройствами, класс Device},
    label=lst:architecture:device
]{src/device.py}

В текущей поставке программной системы реализован класс SoftwareArbiter, который представляет собой обертку над программно-симулированным устройством с PUF типа арбитр. Объект класса SoftwareArbiter инициализируется заранее рассчитанными значениями задержек на узлах PUF. Для использования SoftwareArbiter, как и любого другого класса, реализующего методы Device, достаточно вызвать метод объекта SoftwareArbiter.f с битовым массивом значений управляющих сигналов в качестве аргумента. Применительно к PUF типа арбитр функция f рассчитывает выходной бит по следующей формуле:

\begin{equation}
  \label{eq:architecture:arbiter}
  \sum_{j=1}^{N}(-1)^{\rho_j}\delta_j + \delta_{N+1}\mathop{\lessgtr}_{r=1}^{r=0} 0\text{\,,}
\end{equation}
\begin{explanation}
где & $ \rho_j $ & количество единичных битов входного сигнала; \\
    & $ N $ & количество звеньев в цепи; \\
    & $ \delta_j $ & значение задержки на каждом звене. \\
\end{explanation}



\subsection{Библиотека протокола аутентификации}
Реализация протокола аутентификации, используемая в разработанном ПО, является частным случаем т.н. протоколов аутентификации на основе поиска подстроки (substring matching-based authentication protocol). Данный тип протоколов представил в 2012 году M. Majzoobi под названием \emph{Slender  PUF  protocol}~\cite{slender_puf}.

Протокол предназначен для использования со стойкими физически неклонируемыми функциями (Strong PUFs), является очень легковесным и отлично подходит для реализации в устройствах и системах с ограниченными вычислительными ресурсами. В отличие от уже устоявшейся парадигмы, данный протокол не подразумевает раскрытие полной последовательности выходных сигналов, даже в зашифрованном или трансформированном каким-либо другим способом виде. Вместо этого, из последовательности выделяются случайные части и отсылаются как доказательство подлинности. Для аутентификации устройства проверяющая сторона использует заранее сгенерированные шаблоны выходной последовательности (response patterns).

При корректной реализации протокол может обеспечить превосходную устойчивость против любых известных атак, использующих построение модели PUF-устройства с помошью методов машинного обучения. В дополнение к этому, в протокол сразу заложена коррекция ошибок, которые могут возникнуть при получении выходной строки по уже рассмотренным причинам. Благодаря этому, не требуется реализация сторонних алгоритмов коррекции ошибок, отнимающих драгоценные ресурсы, как не требуется и реализация хэш-функций, нечетких экстракторов (fuzzy extractors), предложенных в ранее известных протоколах аутентификации на основе ФНФ.

В основе протокола лежит построение компактной модели PUF методом машинного обучения. При этом оговаривается, что обучение модели должно быть доступно только на этапе регистрации, после чего возможность извлечения полных выходных последовательностей должна быть заблокировано физическим путем.

Библиотека протокола аутентификации представляет собой небольшой набор классов и функций, реализующих протокол аутентификации PUF на основе поиска подстроки. Сама схема протокола аутентификации продемонстрирована на рисунке \ref{fig:architecture:protocol}.

% \afterpage{
\begin{figure}[!h]
    \centering
    \includegraphics[width=0.93\textwidth]{protocol.png}
    \caption{Схема протокола аутентификации}
    \label{fig:architecture:protocol}
\end{figure}
% }

\clearpage
Набор функций, реализованных в библиотеке, напрямую определяется требованиями протокола. Библиотека предоставляет следующий функционал, используемый как клиентской, так и серверной частью:
\begin{itemize}
  \item Класс-обёртка TRNG для генерации истинно случайных значений с использованием системного генератора случайных чисел. Например в ОС Linux таковым является специальное устройство \emph{/dev/urandom}, предоставляющее интерфейс к пулу случайной информации, сгенерированной драйверами устройств путём сбора метрик шумовых процессов, протекающих при их работе. Применительно к протоколу TRNG используется для получения случайного зерна для PRNG и случайного индекса выходной подпоследовательности в режиме аутентификации и генерации запросов \emph{Challenge} для обучения модели в режиме регистрации.
  \item Класс-обёртка PRNG для генерации псевдослучайных значений. Использует реализацию алгоритма вихря Мерсенна из стандартной библиотеки языка Python. Требует инициализации с помощью зерна. PRNG используется для генерации запросов \emph{Challenge} в режиме аутентификации.
  \item Абстрактный класс Party для обозначения участников взаимодействия по протоколу. Предоставляет реализующим его классам функционал, общий для всех участников: методы идентификации в системе, методы обмена частями случайного зерна, методы генерации псевдослучайных последовательностей входных запросов.
  \item Класс Verifier, реализующий абстрактный класс Party. Представляет сущность сервера проверки подлинности и содержит методы для этапов протокола, специфичных для проверяющей стороны.
  \item Класс Prover, реализующий абстрактный класс Party. Представляет сущность клиентского приложения, представляющего устройство и содержит методы для этапов протокола, специфичных для стороны, запрашивающей проверку.
  \item Класс Configuration, представляющий собой набор конфигурационных параметров протокола и поддерживает инициализацию с помощью файла настроек или системных переменных окружения в дополнение к инициализации с помощью стандартных параметров конструктора.
\end{itemize}

Класс Configuration разобран чуть подробнее, так как от правильного его использования зависит применение параметров работы протокола, влияющих главным образом, на процент ложных срабатываний или наоборот, ложных отказов.
\clearpage
\lstinputlisting[
    style=commonstyle,
    firstline=5,
    caption={Класс Configuration},
    label=lst:architecture:configuration
]{src/configuration.py}

В таблице \ref{table:architecture:cfg_items} приведены краткие описания доступных параметров:

\begin{table}[ht]
  \caption{Параметры протокола аутентификации}
  \label{table:architecture:cfg_items}
  \begin{tabular}{| >{\raggedright}m{0.45\textwidth}
                  | >{\raggedright\arraybackslash}m{0.5\textwidth}|}
   \hline
   Параметр & Описание
   \\ \hline
   \textit{NONCE\_SIZE} & Длина (в битах) половины случайного зерна (результирующее зерно будет иметь длину \textit{2xNONCE\_SIZE}), используемого для инициализации ГПСЧ
   \\ \hline
   \textit{SUBSTR\_LEN} & Длина (в битах) извлекаемой из ответа реального устройства подстроки
   \\ \hline
   \textit{THRESHOLD} & Максимальное значение расстояния Хэмминга между строками, при котором они всё еще считаются схожими
   \\ \hline
   \textit{RSP\_LEN} & Фиксированная длина (в битах) ответа устройства (используется только для тестирования!)
   \\ \hline
  \end{tabular}
\end{table}

В следующих подразделах будут рассмотрены две стороны, взаимодействующие в рамках протокола.

\subsection{Модуль контроля доступа (клиентская часть)}
Так как протокол аутентификации по сути является набором правил, регулирующих обмен информацией нескольких сторон (в данном случае двух), логично описать правила поведения каждой из сторон по отдельности. В реализуемом протоколе участвуют \emph{клиент} -- сторона, имеющая доступ к устройству и <<представляющая его интересы>> в процессе аутентификации и \emph{сервер} -- сторона, обладающая подтверждение подлинности устройства.

В данной упрощённой реализации клиент также контролирует доступ к ресурсу, запрашиваемому устройством. Простейший пример -- электронный ключ в виде устройства с PUF запрашивает доступ к некоторому ресурсу, защищённому электронным замком. Электронный замок ничего не знает о подлинности устройства, в свою очередь устройство не может в одиночку доказать свою подлинность, т.к. по умолчанию не является доверенным. Здесь устройство пользуется <<услугой>> клиента (Prover), который является посредником между устройством, доказывающим свою подлинность, и доверенным сервером, обладающим релевантной информацией насчёт неё. Посредник также контролирует электронный замок, т.е. контролирует, какие устройства могут иметь к секретному ресурсу по результатам процесса аутентификации. Совмещение ролей посредника и барьера, конечно же, не является обязательным и сделано для упрощения архитектуры программного средства и эти роли могут быть без особых проблем разделены для более тонкого контроля над всей инфраструктурой.

По правилам протокола аутентификации клиентский модуль выполняет следующие функции:
\begin{itemize}
  \item генерировать истинно случайные и псевдослучайные данные для использования в алгоритмах протокола;
  \item взаимодействовать с устройством, посылая наборы входных сигналов и считывая результаты на выходе;
  \item создавать компактную модель уPUF на этапе регистрации;
  \item взаимодействовать с сервером, обмениваясь с ним секретной информацией -- с помощью библиотеки requests и грамотного использования защищенных соединений  и клиентских сессий.
\end{itemize}

С добавлением роли барьера примешиваются дополнительно функции надёжного хранения ссылки на секретный ресурс и предоставления устройству ссылки на секретный ресурс  в случае успешной аутентификации.

Как можно увидеть, модуль клиентской стороны использует множество ранее описанных классов и функций, не описывая своих классов или структур данных. Основной задачей клиентского модуля является реализация алгоритма проведения операций регистрации и атуентификации с использованием уже созданных инструментов. В таблице~\ref{table:architecture:client_funcs} описано, как реализованы эти инструменты, на основе которых построен алгоритм, являющийся главной логикой клиентского модуля.

\begin{table}[ht]
  \caption{Функции клиентского модуля и их реализации}
  \label{table:architecture:client_funcs}
  \begin{tabular}{| >{\raggedright}m{0.45\textwidth}
                  | >{\raggedright\arraybackslash}m{0.5\textwidth}|}
   \hline
   Функция & С помощью чего реализована
   \\ \hline
   Генерация истинно случайных чисел & Класс TRNG модуля протокола аутентификации
   \\ \hline
   Генерация псевдослучайных чисел & Класс PRNG модуля протокола аутентификации
   \\ \hline
   Обмен битовыми строками входных и выходных сигналов с утройствами & Класс Device и его производные классы в слое взаимодействия с устройствами
   \\ \hline
   Создание компактной модели PUF & Класс-обёртка над модулем построения модели (для обеспечения интероперабельности языков Python и Haskell)
   \\ \hline
   Контроль доступа у секретным ресурсам & Внутренние функции модуля или опциональный внешний интерфейс.
   \\ \hline
  \end{tabular}
\end{table}

\subsection{Модуль проверки подлинности (серверная часть)}
В описанном выше протоколе \emph{сервер} является сущностью, располагающей информацией о подлинности зарегистрированных устройств. Эта информация представлена в виде компактных моделей PUF, созданных на этапе их регистрации.

Функционал сервера во многом повторяет таковой у клиента, именно поэтому было принято решение вынести общие функции в абстрактный класс Prover. Сервер использует свою реализацию этого класса -- Verifier.

На основании вышесказанного, серверный модуль проверки подлинности выполняет следующие функции:
\begin{itemize}
  \item генерировать истинно случайные и псевдослучайные данные для использования в алгоритмах протокола;
  \item взаимодействовать с моделью устройства, посылая наборы входных сигналов и считывая результаты на выходе;
  \item иметь доступ к хранилищу моделей PUF;
  \item использовать алгоритм нечеткого поиска строки для сравнения данных, полученных от реального устройства с данными, полученными от компактной модели;
  \item работать в режиме сервера, т.е. принимать и обрабатывать входящие запросы, все время находясь в ожидании новых запросов;
\end{itemize}


\lstinputlisting[
    style=commonstyle,
    firstline=18,
    caption={Метод нечеткого поиска подстроки с заданным порогом},
    label=lst:architecture:fuzzysearch
]{src/fulllisting/puf.py}

\begin{table}[ht]
  \caption{Функции серверного модуля и их реализации}
  \label{table:architecture:client_funcs}
  \begin{tabular}{| >{\raggedright}m{0.45\textwidth}
                  | >{\raggedright\arraybackslash}m{0.5\textwidth}|}
   \hline
   Функция & С помощью чего реализована
   \\ \hline
   Работа в качестве HTTP-сервера по защищенному протоколу & Веб-фреймворка Flask и его поддержка безопасных соединений и клиентских сессий.
   \\ \hline
   Генерация истинно случайных чисел & Класс TRNG модуля протокола аутентификации
   \\ \hline
   Генерация псевдослучайных чисел & Класс PRNG модуля протокола аутентификации
   \\ \hline
   Обмен битовыми строками входных и выходных сигналов с моделью PUF & Класс Model
   \\ \hline
   Работа с хранищилем моделей PUF & Класс Database и его реализации для работы с файловой системой, Redis или MongoDB
   \\ \hline
   Нечеткий поиск подстроки & Класс Response модуля PUF и его реализации нечеткого поиска
   \\ \hline
  \end{tabular}
\end{table}


Сервер проверки подлинности должен иметь возможность обмена данными с моделью образом, похожим на способ взаимодействия с устройствами для модуля контроля доступа. Для данной цели создан класс Model, который предоставляет такой же интерфейс к модели PUF, какой предоставляет слой взаимодействия с устройствами для реальных PUF. Метод \emph{f} класса Model принимает масиив битов класса Challenge и возвращает бит ответа -- объект класса Response.

Для взаимодействия с хранилищем моделей модуль проверки подлинности предоставляет простой класс Database, по умолчанию реализующий хранение моделей в виде файлов на файловой системе. В то время как такой подход может быть полезен при тестировании приложения, для реального использования стоит также рассмотреть другие варианты:
\begin{itemize}
  \item NoSQL база данных. Так как модели PUF являются независимыми наборами векторов, использование классической реляционной базы данных в данном случае нерационально, ведь модели не имеют множества атрибутов, как не имеют и связей и отношений с другими сущностями в БД. В рамках этого варианта программное средство поддерживает работу с нереляционной СУБД MongoDB.
  \item Кэш (неперсистентное хранилище). В зависимости от требований информационной системы, в которой планируется использование данного ПС, является вероятным сценарий того, что модели PUF не должны храниться в базе слишком долго, или даже могут являться одноразовыми. В данном случае можно использовать систему хранения временных данных в памяти (memcached, Redis). Более того, так как в системе принято условие <<одно устройство -- одна модель>>, можно обойтись простейшим хранилищем типа <<ключ-значение>>(\emph{key-value store}). Для этого варианта реализован класс для работы с хранилищем Redis.
\end{itemize}
 %

% Глава 4 Тестирование программного средства
\section{Тестирование программного средства}
\label{sec:testing}

Программное средство разрабатывалось в виде набора библиотек. Для использования программной системы необходимо взаимодействие с внешними устройствами, и драйвер, реализуюший этот функционал. Тестирование приложеня предполагает проверку работоспособности самой библиотеки независимо от используемого драйвера. Поэтому перед началом тестирования необходимо произвести следующие подготовительные действия:
\begin{enumerate}
  \item корректно сконфигурировать серверную часть приложения;
  \item на клиентской стороне подключить необходимые библиотеки из состава программной системы;
  \item инициализировать слой взаимодействия с устройствами;
  \item инициировать протокол регистрации или аутентификации устройства в системе.
\end{enumerate}

Тестирование предполагает наличие файлов конфигурации и файла инициализации. В качестве устройства-субъекта аутентификации на этапе тестирования используется программная симуляция ФНФ для обеспечения однозначости и повторяемости результатов тестов.

Для оценки правильности работы программного средства было проведено тестирование. В данном разделе приведены тест-кейсы и их результаты в вид таблиц.

\begin{longtable}[l]{| >{\raggedright}p{0.3\textwidth}
                     | >{\raggedright}p{0.3\textwidth}
                     | >{\raggedright\arraybackslash}p{0.3\textwidth}|}
  \caption{Тестирование доступа к базе моделей устройств}
  \label{table:testing:db}\\
  \endfirsthead
  \caption*{Продолжение таблицы \ref{table:testing:db}}\\
  \endhead

  \hline
       Название тест-кейса и его описание & Ожидаемый результат & Фактический результат \\
   \hline
   Получение модели PUF по id устройства \\
   1) Инициализировать объект класса Database; \\
   2) вызвать функцию Database.get с id зарегистрированного устройства в качестве параметра.
   &
   1) Объект класса Database сигнализирует об успешной инициализации; \\
   2) результатом функции является объект класса Model с параметром id, использованным в запросе.
   &
   Тест пройден \\ \hline

\end{longtable}

\begin{longtable}[l]{| >{\raggedright}p{0.3\textwidth}
                     | >{\raggedright}p{0.3\textwidth}
                     | >{\raggedright\arraybackslash}p{0.3\textwidth}|}
  \caption{Тестирование программы построения компактной модели PUF}
  \label{table:testing:bayes}\\
  \endfirsthead
  \caption*{Продолжение таблицы \ref{table:testing:bayes}}\\
  \endhead

  \hline
       Название тест-кейса и его описание & Ожидаемый результат & Фактический результат \\
   \hline
   Запуск обучения модели из файла \\
   1) В интерфейсе коммандной строки набрать путь исполняемого файла и путём к файлу данных; \\
   2) нажать клавишу ввода и ожидать завершения программы.
   &
   1) Имя исполняемого модуля и путь к файлу отображаются в консоли; \\
   2) отображается результат обучения модели в виде вектора апостериорных вероятностей.
   &
   Тест пройден \\ \hline

   Запуск обучения модели из файла с заданным количеством попыток \\
   1) В интерфейсе коммандной строки набрать путь исполняемого файла модуля с путём к файлу векторов данных и числом попыток обучения модели в качестве аргументов; \\
   2) нажать клавишу ввода и ожидать завершения программы.
   &
   1) Имя исполняемого модуля, путь к файлу и число попыток обучения отображаются в консоли; \\
   2) отображаются обученные модели в количестве, равном числу попыток, и отмечается модель с наименьшим процентом ошибок.
   &
   Тест пройден \\

   \hline
\end{longtable}

Далее тестируется запуск сервера с заданной конфигурацией? котороя может быть представлена в виде ini-файла или с описана с помощью перемеженных окружения.
\clearpage

\begin{longtable}[l]{| >{\raggedright}p{0.3\textwidth}
                     | >{\raggedright}p{0.3\textwidth}
                     | >{\raggedright\arraybackslash}p{0.3\textwidth}|}
  \caption{Тестирование программы построения компактной модели PUF}
  \label{table:testing:servercfg}\\
  \endfirsthead
  \caption*{Продолжение таблицы \ref{table:testing:servercfg}}\\

  \hline
       Название тест-кейса и его описание & Ожидаемый результат & Фактический результат \\
  \endhead
   \hline
   Запуск сервера с настройками по умолчанию \\
   1) В коммандной строки набрать команду запуска сервера; \\
   2) нажать клавишу ввода.
   &
   1) Имя скрипта запуска сервера отображаются в консоли; \\
   2) отображается диагностическая информация о запуске сервера и предустановленных параметра функционирования.
   &
   Тест пройден \\ \hline

   Запуск сервера с настройками из ini-файла \\
   1) В коммандной строки набрать команду запуска сервера и путь к ini-файлу в качестве аргумента; \\
   2) нажать клавишу ввода.
   &
   1) Имя скрипта запуска сервера отображаются в консоли; \\
   2) отображается диагностическая информация о запуске сервера и параметра функционирования, взятых из преоставленного ini-файла.
   &
   Тест пройден \\ \hline
\end{longtable}

\clearpage
\begin{longtable}[l]{| >{\raggedright}p{0.3\textwidth}
                     | >{\raggedright}p{0.3\textwidth}
                     | >{\raggedright\arraybackslash}p{0.3\textwidth}|}
  \caption{Тестирование взаимодействия с уcтройством}
  \label{table:testing:pufbit}\\
  \endfirsthead
  \caption*{Продолжение таблицы \ref{table:testing:pufbit}}\\
  \endhead

  \hline
       Название тест-кейса и его описание & Ожидаемый результат & Фактический результат \\
   \hline
   Использование функции PUF для получения выходного бита \\
   1) Инициировать экземпляр класса Prover c объектом Device в качестве аргумента; \\
   2) сгенерировать случайный объект Challenge с помощью объекта Prover; \\
   3) запустить функцию Device.f с аргументом Challenge.
   &
   1) Объект класса Prover возвращает статус успешной инициализации; \\
   2) объект Challenge является битовой строкой из случайных нулей и единиц; \\
   3) значением функции является случайный бит 0 или 1.
   &
   Тест пройден \\ \hline

   Использование модели функции PUF для получения выходного бита \\
   1) Инициировать экземпляр класса Verifier c объектом Device в качестве аргумента; \\
   2) сгенерировать случайный объект Challenge с помощью объекта Verifier; \\
   3) запустить функцию Model.f с аргументом Challenge.
   &
   1) Объект класса Verifier возвращае статус успешной инициализации; \\
   2) объект Challenge является битовой строкой из случайных нулей и единиц; \\
   3) значением функции является случайный бит 0 или 1.
   &
   Тест пройден \\ \hline

   \hline
\end{longtable}

\clearpage

В следующих таблицах(\ref{table:testing:proto} и \ref{table:testing:regpuf}) тестируются два основных режима: режим регистрации и режим аутенификации, при тестировании которого проверятся также случай, когда злоумышленником прелпринимается попытка аутентификации с помощью поддельного PUF. Даже случае подмены идентификатора устройства решение о предоставлении доступа будет принято на основе соответствия поведения устройства поведению зарегистрированной модели.


\begin{longtable}[l]{| >{\raggedright}p{0.3\textwidth}
                     | >{\raggedright}p{0.3\textwidth}
                     | >{\raggedright\arraybackslash}p{0.3\textwidth}|}
  \caption{Тестирование протокола аутентификации}
  \label{table:testing:proto}\\
    \hline
       Название тест-кейса и его описание & Ожидаемый результат & Фактический результат \\
    \hline
  \endfirsthead
  \caption*{Продолжение таблицы \ref{table:testing:proto}}\\

  \hline
       Название тест-кейса и его описание & Ожидаемый результат & Фактический результат \\
   \hline
  \endhead
  \endlastfoot
   Аутентификация незарегистрированного устройства \\
   1) Инициировать объект класса Device, связанного с незарегистрированным устройством; \\
   2) подменить поле id объекта Device, на id, связанный с зарегистрированным устройством; \\
   3) инициировать экземпляр класса Prover c объектом Device и адресом сервера аутентификации в качестве аргументов; \\
   4) вызвать функцию Prover.authenticate.
   &
   1) Объект класса Device имеет id незарегистрированного устройства; \\
   2) объект класса Device имеет id зарегистрированного устройства; \\
   3) объект класса Prover возвращает статус успешной инициализации; \\
   4) сервер возвращает сообщение об отклонение запроса и предупреждение о поддельном устройстве.
   &
   Тест пройден \\

   Аутентификация подлинного устройства \\
   1) Инициировать объект класса Device, связанного с зарегистрированным устройством; \\
   2) инициировать экземпляр класса Prover c объектом Device и адресом сервера аутентификации в качестве аргументов; \\
   3) вызвать функцию Prover.authenticate.
   &
   1) Объект класса Device имеет id зарегистрированного устройства; \\
   2) объект класса Prover возвращает статус успешной инициализации; \\
   3) сервер возвращает сообщение об успешной аутентификации.
   &
   Тест пройден \\
   \hline


\end{longtable}


\begin{longtable}[l]{| >{\raggedright}p{0.3\textwidth}
                     | >{\raggedright}p{0.3\textwidth}
                     | >{\raggedright\arraybackslash}p{0.3\textwidth}|}
  \caption{Тестирование регистрации устройства}
  \label{table:testing:regpuf}\\
  \endfirsthead
  \caption*{Продолжение таблицы \ref{table:testing:regpuf}}\\
  \endhead

  \hline
       Название тест-кейса и его описание & Ожидаемый результат & Фактический результат \\
   \hline
   Регистрация устройства в базе данных \\
   1) Инициировать экземпляр класса Prover c объектом Device в качестве аргумента; \\
   2) вызвать функцию Prover.register. \\
   &
   1) Объект класса Prover возвращает статус успешной инициализации; \\
   2) сервер возвращает сообщение об успешной регистрации.
   &
   Тест пройден \\ \hline
\end{longtable}
 %

% Глава 5 Использование разработанного программного средства
\section{Методика использования программной системы}
\label{sec:manpage}

Программное средство распознавания уникальных неклонируемых идентификаторов цифровых устройств является комплексным продуктом и рассчитано на работу по схеме клиент-сервер. Ниже приведен возможный вариант развертывания и конфигурации программной системы.

\subsection{Настройка сервера}
\label{sec:manpage:server_setup}
Модуль проверки подлинности (сервер) может запускаться в одиночном режиме как обычное Python-приложение. Однако, такой подход не является рекомендованным и не обеспечивает достаточную секретность информации и стабильность системы. Сервер аутентификации рассчитан на работу в связке с WSGI-сервером и веб-сервером. Возможная схема их подключения показана на рисунке \ref{fig:manpage:nginx_proxy}.

\begin{figure}[!h]
    \centering
    \includegraphics[width=0.65\textwidth]{nginx_proxy.png}
    \caption{Схема возможного развертывания приложения в связке с WSGI-сервером и HTTP-сервером}
    \label{fig:manpage:nginx_proxy}
\end{figure}


Для обеспечения лучшей совместимости программных компонентов рекомендуется развертывание на устройствах под управением операционной системы семейства Linux. В листингах \ref{lst:manpage:gunicorncfg} и \ref{lst:manpage:nginxcfg} приведен пример рекомендуемой конфигурации менеджера процессов \emph{systemd} для запуска WSGI-сервера \emph{gunicorn} и настроек HTTP-сервера \emph{nginx} соответственно.


Программное средство рассчитано на работу по протоколу HTTPS в целях обеспечения секретности передаваемой информации. Для развертывания внутри закрытой сети достаточно использования самоподписанного (\emph{self-signed}) SSL/TLS"=сертификата, однако, если сценарий использования предполагает прохождение информационных потоков через публичные сети, например, Интернет, крайне рекомендуется использование валидного сертификата, подписанного доверенным центром сертификации. В обоих случаях необходимо правильная настройка сетевого экрана, для исключения перехвата данных или осуществления атаки отказа в обслуживании (DDoS) злоумышленником. При использовании вышеописанного сценария развертывания, достаточно разрешения входящего и исходящего трафика по стандартному HTTPS-порту 443.


\lstinputlisting[
    style=commonstyle,
    caption=Пример конфигурации systemd для запуска приложения в связке с WSGI-сервером gunicorn,
    label=lst:manpage:gunicorncfg
]{src/systemdgunicorn.conf}

\lstinputlisting[
    style=commonstyle,
    caption=Пример конфигурации nginx в качестве прокси-сервера,
    label=lst:manpage:nginxcfg
]{src/nginx.conf}

В целом, настройка и развертывание серверной части ничем не отличается от развертывания любого другого Python/WSGI-приложения, поэтому связанные с ними нюансы не являются частью данного руководства.

\subsection{Настройка модуля контроля доступа}
\label{sec:manpage:client_setup}
Модуль контроля доступа (Prover) является посредником между устройством и сервером аутентификации. Он не хранит секретных данных, однако может взаимодействать с реальными устройствами, поэтому он также должен быть развернут в безопасном окружении. Модуль контроля доступа сообщается с модулем проверки подлинности по протоколу HTTPS, как показано на рисунке \ref{fig:manpage:nginx_proxy}.


\subsection{Пример использования}
Программное средство поддерживает два этапа взаимодействия устройства с информационной системой -- регистрацию и аутентификацию. Для обеих целей в состав ПС входят утилиты \emph{register\_device.py} и \emph{authenticate\_device.py}. Они предоставляют базовый функционал и носят скорее демонстрационный характер, так как ПС рассчитана на интеграцию с существующей системой безопасности и предоставляет API для реализации нужд этой системы. Исходный код утилит представлен в листингах \ref{lst:manpage:authdev} и \ref{lst:manpage:regdev}.

\lstinputlisting[
    style=commonstyle,
    caption=Вспомогательный скрипт аутентификации утсройства,
    label=lst:manpage:authdev
]{src/authenticate_device.py}

\lstinputlisting[
    style=commonstyle,
    caption=Вспомогательный скрипт регистрации утсройства,
    label=lst:manpage:regdev
]{src/register_device.py}

\subsection{Расширение функциональности программного средства}
Программное средство может быть расширено для обеспечения поддержки различных типов устройств и подключений. Базовая поставка программной системы включает программную симуляцию PUF-устройства типа арбитр и класс (драйвер) для работы с ним. Добавление новых драйверов происходит путем переопределения базового класса devices.Device и реализации метода создания виртуального объекта устройства и собственного метода функции PUF, предназначенного для передачи входного сигнала устройству и получения выходного.

\lstinputlisting[
    style=commonstyle,
    caption={Пример переопределения класса Device для работы с программной реализацией PUF-арбитра},
    label=lst:architecture:software_arbiter
]{src/fulllisting/software_arbiter.py}
 %

% Глава 6 Технико-экономическое обоснование разработки ПС
\input{tex/economics} %

% Заключение
\sectioncentered*{Заключение}
\addcontentsline{toc}{section}{Заключение}
\label{sec:outro}

В данном дипломном проекте был рассмотрен вопрос использования физически неклонируемых функций в качестве субъектов аутентификации. В рамках дипломного проекта была разработана программное средство с клиент-серверной архитектурой, предоставляющее возможности для удалённой проверки подлинности цифровых устройств и рассчитанное на интеграцию в новые или существующие системы информационной или физической безопасности. В разработанном проекте были использованы современные теоретические наработки, технологии машинного обучения и статистического моделирования, а также современные программные инструменты и языки программирования. Реализованные в проекте функций обеспечивают достаточный уровень секретности передаваемых данных, а также отвечают высоким требованиям производительности и стабильности программного средства.

В итоге, тема дипломного проекта была раскрыта, а в его рамках создано комплексное многофункциональное программное средство. Однако, за рамками рассматриваемой темы осталось множество альтернативных протоколов аутентификации и их реализаций, а также аппаратная реализация физически неклонируемой функции и алгоритмов, использованных в протоколе. В дальнейшем планируется как можно более приблизить программное средство к использованию с реальными цифровыми устройствами для решения задач не только проверки подлинности, но и смежных, таких как защита интеллектуальной собственности.
 %

% Список использованных источников
\input{tex/references} %

% % Приложение А Исходный код программного средства
\sectioncentered*{ПРИЛОЖЕНИЕ А}
\label{sec:sources}
\addcontentsline{toc}{section}{Приложение А Исходный код программного средства}
\begin{center}
\vspace{-1em}
\textbf{ (обязательное)}

\textbf{Исходный код программного средства}
\end{center}

\lstinputlisting[style=commonstyle]{src/fulllisting/Main.hs}
\lstinputlisting[style=commonstyle]{src/fulllisting/Options.hs}
\lstinputlisting[style=commonstyle]{src/fulllisting/InputUtils.hs}
\lstinputlisting[style=commonstyle]{src/fulllisting/Statistics.hs}
\lstinputlisting[style=commonstyle]{src/fulllisting/NaiveBayes.hs}

\lstinputlisting[style=commonstyle]{src/fulllisting/device.py}
\lstinputlisting[style=commonstyle]{src/fulllisting/software_arbiter.py}
\lstinputlisting[style=commonstyle]{src/fulllisting/1234.deltas}
\lstinputlisting[style=commonstyle]{src/fulllisting/puf.py}
\lstinputlisting[style=commonstyle]{src/fulllisting/protocol.py}
\lstinputlisting[style=commonstyle]{src/fulllisting/prover.py}
\lstinputlisting[style=commonstyle]{src/fulllisting/verifier.py}
\lstinputlisting[style=commonstyle]{src/fulllisting/database.py}
\lstinputlisting[style=commonstyle]{src/fulllisting/milk.py}
\lstinputlisting[style=commonstyle]{src/fulllisting/runserver.py}
\lstinputlisting[style=commonstyle]{src/authenticate_device.py}
\lstinputlisting[style=commonstyle]{src/register_device.py}
\lstinputlisting[style=commonstyle]{src/fulllisting/abstractsimulator.py}
\lstinputlisting[style=commonstyle]{src/fulllisting/ropuf.py}
 %

\end{document}
