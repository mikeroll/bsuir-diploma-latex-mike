\sectioncentered*{Введение}
\addcontentsline{toc}{section}{Введение}
\label{sec:intro}

Физическая криптография, основанная на структурной сложности физических систем, является одним из наиболее современных достижений в области криптографии и защиты информации. Данное направление в области защиты информации является значительным шагом вперед по сравнению с классической (алгебраической) криптографии, опирающейся только на математические методы и предоставляет качественно новые средства генерации и обработки информации, так как строится на основе эксплуатации шумоподобного поведения физических объектов и систем. Стойкость системы к статистическому криптоанализу в физической криптографии выведена на новый уровень, в основном благодаря истинной, плохо поддающейся математическому моделированию случайности значений параметров, извлекаемых из физической системы.

Главной сущностью физической криптографии является физически неклонируемая функция (ФНФ) -- физическая система, обладающая множеством компонент, параметры которых в процессе создания подобных физических систем принимают случайные значения. Главным свойством ФНФ является собственно неклонируемость -- безусловная невозможность создания точной копии строения или поведения этой системы.

В рамках данного дипломного проекта рассматриваются методы использования ФНФ в качестве уникальных идентификаторов физических сущностей, в частности -- электронных устройств. Логическим продолжением процесса идентификации сущности является аутентификация -- т.е. проверка подлинности предъявленного идентификатора. В представленном проекте изучены методы построения систем аутентификации таких идентификаторов, реализован легковесный протокол, решающий эту задачу, и программная система обмена секретной информацией по этому протоколу, которая может использоваться в качестве основы для построения систем информационной или физической безопасности.
