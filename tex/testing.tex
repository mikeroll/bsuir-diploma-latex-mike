\section{Тестирование программного средства}
\label{sec:testing}

Программное средство разрабатывалось в виде набора библиотек. Для использования программной системы необходимо взаимодействие с внешними устройствами, и драйвер, реализуюший этот функционал. Тестирование приложеня предполагает проверку работоспособности самой библиотеки независимо от используемого драйвера. Поэтому перед началом тестирования необходимо произвести следующие подготовительные действия:
\begin{enumerate}
  \item корректно сконфигурировать серверную часть приложения;
  \item на клиентской стороне подключить необходимые библиотеки из состава программной системы;
  \item инициализировать слой взаимодействия с устройствами;
  \item инициировать протокол регистрации или аутентификации устройства в системе.
\end{enumerate}

Тестирование предполагает наличие файлов конфигурации и файла инициализации. В качестве устройства-субъекта аутентификации на этапе тестирования используется программная симуляция ФНФ для обеспечения однозначости и повторяемости результатов тестов.

Для оценки правильности работы программного средства было проведено тестирование. В данном разделе приведены тест-кейсы и их результаты в вид таблиц.

\begin{longtable}[l]{| >{\raggedright}p{0.3\textwidth}
                     | >{\raggedright}p{0.3\textwidth}
                     | >{\raggedright\arraybackslash}p{0.3\textwidth}|}
  \caption{Тестирование доступа к базе моделей устройств}
  \label{table:testing:db}\\
  \endfirsthead
  \caption*{Продолжение таблицы \ref{table:testing:db}}\\
  \endhead

  \hline
       Название тест-кейса и его описание & Ожидаемый результат & Фактический результат \\
   \hline
   Получение модели PUF по id устройства \\
   1) Инициализировать объект класса Database; \\
   2) вызвать функцию Database.get с id зарегистрированного устройства в качестве параметра.
   &
   1) Объект класса Database сигнализирует об успешной инициализации; \\
   2) результатом функции является объект класса Model с параметром id, использованным в запросе.
   &
   Тест пройден \\ \hline

\end{longtable}

\begin{longtable}[l]{| >{\raggedright}p{0.3\textwidth}
                     | >{\raggedright}p{0.3\textwidth}
                     | >{\raggedright\arraybackslash}p{0.3\textwidth}|}
  \caption{Тестирование программы построения компактной модели PUF}
  \label{table:testing:bayes}\\
  \endfirsthead
  \caption*{Продолжение таблицы \ref{table:testing:bayes}}\\
  \endhead

  \hline
       Название тест-кейса и его описание & Ожидаемый результат & Фактический результат \\
   \hline
   Запуск обучения модели из файла \\
   1) В интерфейсе коммандной строки набрать путь исполняемого файла и путём к файлу данных; \\
   2) нажать клавишу ввода и ожидать завершения программы.
   &
   1) Имя исполняемого модуля и путь к файлу отображаются в консоли; \\
   2) отображается результат обучения модели в виде вектора апостериорных вероятностей.
   &
   Тест пройден \\ \hline

   Запуск обучения модели из файла с заданным количеством попыток \\
   1) В интерфейсе коммандной строки набрать путь исполняемого файла модуля с путём к файлу векторов данных и числом попыток обучения модели в качестве аргументов; \\
   2) нажать клавишу ввода и ожидать завершения программы.
   &
   1) Имя исполняемого модуля, путь к файлу и число попыток обучения отображаются в консоли; \\
   2) отображаются обученные модели в количестве, равном числу попыток, и отмечается модель с наименьшим процентом ошибок.
   &
   Тест пройден \\

   \hline
\end{longtable}

Далее тестируется запуск сервера с заданной конфигурацией? котороя может быть представлена в виде ini-файла или с описана с помощью перемеженных окружения.
\clearpage

\begin{longtable}[l]{| >{\raggedright}p{0.3\textwidth}
                     | >{\raggedright}p{0.3\textwidth}
                     | >{\raggedright\arraybackslash}p{0.3\textwidth}|}
  \caption{Тестирование программы построения компактной модели PUF}
  \label{table:testing:servercfg}\\
  \endfirsthead
  \caption*{Продолжение таблицы \ref{table:testing:servercfg}}\\

  \hline
       Название тест-кейса и его описание & Ожидаемый результат & Фактический результат \\
  \endhead
   \hline
   Запуск сервера с настройками по умолчанию \\
   1) В коммандной строки набрать команду запуска сервера; \\
   2) нажать клавишу ввода.
   &
   1) Имя скрипта запуска сервера отображаются в консоли; \\
   2) отображается диагностическая информация о запуске сервера и предустановленных параметра функционирования.
   &
   Тест пройден \\ \hline

   Запуск сервера с настройками из ini-файла \\
   1) В коммандной строки набрать команду запуска сервера и путь к ini-файлу в качестве аргумента; \\
   2) нажать клавишу ввода.
   &
   1) Имя скрипта запуска сервера отображаются в консоли; \\
   2) отображается диагностическая информация о запуске сервера и параметра функционирования, взятых из преоставленного ini-файла.
   &
   Тест пройден \\ \hline
\end{longtable}

\clearpage
\begin{longtable}[l]{| >{\raggedright}p{0.3\textwidth}
                     | >{\raggedright}p{0.3\textwidth}
                     | >{\raggedright\arraybackslash}p{0.3\textwidth}|}
  \caption{Тестирование взаимодействия с уcтройством}
  \label{table:testing:pufbit}\\
  \endfirsthead
  \caption*{Продолжение таблицы \ref{table:testing:pufbit}}\\
  \endhead

  \hline
       Название тест-кейса и его описание & Ожидаемый результат & Фактический результат \\
   \hline
   Использование функции PUF для получения выходного бита \\
   1) Инициировать экземпляр класса Prover c объектом Device в качестве аргумента; \\
   2) сгенерировать случайный объект Challenge с помощью объекта Prover; \\
   3) запустить функцию Device.f с аргументом Challenge.
   &
   1) Объект класса Prover возвращает статус успешной инициализации; \\
   2) объект Challenge является битовой строкой из случайных нулей и единиц; \\
   3) значением функции является случайный бит 0 или 1.
   &
   Тест пройден \\ \hline

   Использование модели функции PUF для получения выходного бита \\
   1) Инициировать экземпляр класса Verifier c объектом Device в качестве аргумента; \\
   2) сгенерировать случайный объект Challenge с помощью объекта Verifier; \\
   3) запустить функцию Model.f с аргументом Challenge.
   &
   1) Объект класса Verifier возвращае статус успешной инициализации; \\
   2) объект Challenge является битовой строкой из случайных нулей и единиц; \\
   3) значением функции является случайный бит 0 или 1.
   &
   Тест пройден \\ \hline

   \hline
\end{longtable}

\clearpage

В следующих таблицах(\ref{table:testing:proto} и \ref{table:testing:regpuf}) тестируются два основных режима: режим регистрации и режим аутенификации, при тестировании которого проверятся также случай, когда злоумышленником прелпринимается попытка аутентификации с помощью поддельного PUF. Даже случае подмены идентификатора устройства решение о предоставлении доступа будет принято на основе соответствия поведения устройства поведению зарегистрированной модели.


\begin{longtable}[l]{| >{\raggedright}p{0.3\textwidth}
                     | >{\raggedright}p{0.3\textwidth}
                     | >{\raggedright\arraybackslash}p{0.3\textwidth}|}
  \caption{Тестирование протокола аутентификации}
  \label{table:testing:proto}\\
    \hline
       Название тест-кейса и его описание & Ожидаемый результат & Фактический результат \\
    \hline
  \endfirsthead
  \caption*{Продолжение таблицы \ref{table:testing:proto}}\\

  \hline
       Название тест-кейса и его описание & Ожидаемый результат & Фактический результат \\
   \hline
  \endhead
  \endlastfoot
   Аутентификация незарегистрированного устройства \\
   1) Инициировать объект класса Device, связанного с незарегистрированным устройством; \\
   2) подменить поле id объекта Device, на id, связанный с зарегистрированным устройством; \\
   3) инициировать экземпляр класса Prover c объектом Device и адресом сервера аутентификации в качестве аргументов; \\
   4) вызвать функцию Prover.authenticate.
   &
   1) Объект класса Device имеет id незарегистрированного устройства; \\
   2) объект класса Device имеет id зарегистрированного устройства; \\
   3) объект класса Prover возвращает статус успешной инициализации; \\
   4) сервер возвращает сообщение об отклонение запроса и предупреждение о поддельном устройстве.
   &
   Тест пройден \\

   Аутентификация подлинного устройства \\
   1) Инициировать объект класса Device, связанного с зарегистрированным устройством; \\
   2) инициировать экземпляр класса Prover c объектом Device и адресом сервера аутентификации в качестве аргументов; \\
   3) вызвать функцию Prover.authenticate.
   &
   1) Объект класса Device имеет id зарегистрированного устройства; \\
   2) объект класса Prover возвращает статус успешной инициализации; \\
   3) сервер возвращает сообщение об успешной аутентификации.
   &
   Тест пройден \\
   \hline


\end{longtable}


\begin{longtable}[l]{| >{\raggedright}p{0.3\textwidth}
                     | >{\raggedright}p{0.3\textwidth}
                     | >{\raggedright\arraybackslash}p{0.3\textwidth}|}
  \caption{Тестирование регистрации устройства}
  \label{table:testing:regpuf}\\
  \endfirsthead
  \caption*{Продолжение таблицы \ref{table:testing:regpuf}}\\
  \endhead

  \hline
       Название тест-кейса и его описание & Ожидаемый результат & Фактический результат \\
   \hline
   Регистрация устройства в базе данных \\
   1) Инициировать экземпляр класса Prover c объектом Device в качестве аргумента; \\
   2) вызвать функцию Prover.register. \\
   &
   1) Объект класса Prover возвращает статус успешной инициализации; \\
   2) сервер возвращает сообщение об успешной регистрации.
   &
   Тест пройден \\ \hline
\end{longtable}
